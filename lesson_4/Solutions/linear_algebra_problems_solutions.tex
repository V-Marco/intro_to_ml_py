\documentclass[11pt, a4paper]{extarticle}
\setlength{\parskip}{0.2em}
%%% Работа с русским языком
\usepackage{cmap}					% поиск в PDF
\usepackage{mathtext} 				% русские буквы в формулах
\usepackage[T2A]{fontenc}			% кодировка
\usepackage[utf8]{inputenc}			% кодировка исходного текста
\usepackage[english,russian]{babel}	% локализация и переносы
\usepackage{mathtools}   % loads »amsmath«
\usepackage{graphicx}
\usepackage{caption}
\usepackage{physics}
\usepackage{subcaption}
\usepackage{tikz}
\usepackage{multicol}
\usepackage{listings}
\usepackage{tabularx}

\usepackage{hyperref}
\hypersetup{				% Гиперссылки
    unicode=true,           % русские буквы в раздела PDF
    pdftitle={Заголовок},   % Заголовок
    pdfauthor={Автор},      % Автор
    pdfsubject={Тема},      % Тема
    pdfcreator={Создатель}, % Создатель
    pdfproducer={Производитель}, % Производитель
    pdfkeywords={keyword1} {key2} {key3}, % Ключевые слова
    colorlinks=true,       	% false: ссылки в рамках; true: цветные ссылки
    linkcolor=black,          % внутренние ссылки
    citecolor=black,        % на библиографию
    filecolor=cyan,      % на файлы
    urlcolor=gray           % на URL
}

\usepackage{setspace} % Интерлиньяж
%\onehalfspacing % Интерлиньяж 1.5
%\doublespacing % Интерлиньяж 2
\singlespacing % Интерлиньяж 1

%%% Дополнительная работа с математикой
\usepackage{amsmath,amsfonts,amssymb,amsthm,mathtools} % AMS
\usepackage{icomma} % "Умная" запятая: $0,2$ --- число, $0, 2$ --- перечисление

%% Шрифты
\usepackage{euscript}	 % Шрифт Евклид
\usepackage{mathrsfs} % Красивый матшрифт
\usepackage{float}

\title{Элементы Линейной Алгебры \\ (Теория и Решения)}

\author{Факультатив «Введение в анализ данных и машинное обучение на Python»}
\date{\today}

\usepackage{geometry}
\geometry{
	a4paper,
	left=20mm,
	top=20mm,
	right=20mm
}
\setlength{\parindent}{0cm}

\DeclareMathOperator{\Lin}{\mathrm{Lin}}
\DeclareMathOperator{\Linp}{\Lin^{\perp}}
\DeclareMathOperator*\plim{plim}
%\DeclareMathOperator{\grad}{grad}
\DeclareMathOperator{\card}{card}
\DeclareMathOperator{\sgn}{sign}
\DeclareMathOperator{\sign}{sign}

\DeclareMathOperator*{\argmin}{arg\,min}
\DeclareMathOperator*{\argmax}{arg\,max}
\DeclareMathOperator*{\amn}{arg\,min}
\DeclareMathOperator*{\amx}{arg\,max}
\DeclareMathOperator{\cov}{Cov}
\DeclareMathOperator{\Var}{Var}
\DeclareMathOperator{\Cov}{Cov}
\DeclareMathOperator{\Corr}{Corr}
\DeclareMathOperator{\pCorr}{pCorr}
\DeclareMathOperator{\E}{\mathbb{E}}
\let\P\relax
\DeclareMathOperator{\P}{\mathbb{P}}



\newcommand{\cN}{\mathcal{N}}
\newcommand{\cU}{\mathcal{U}}
\newcommand{\cBinom}{\mathcal{Binom}}
\newcommand{\cPois}{\mathcal{Pois}}
\newcommand{\cBeta}{\mathcal{Beta}}
\newcommand{\cGamma}{\mathcal{Gamma}}

\def \R{\mathbb{R}}
\def \N{\mathbb{N}}
\def \Z{\mathbb{Z}}

\usepackage{multirow}
\usepackage{array}


\newcommand{\dx}[1]{\,\mathrm{d}#1} % для интеграла: маленький отступ и прямая d
\newcommand{\ind}[1]{\mathbbm{1}_{\{#1\}}} % Индикатор события
%\renewcommand{\to}{\rightarrow}
\newcommand{\eqdef}{\mathrel{\stackrel{\rm def}=}}
\newcommand{\iid}{\mathrel{\stackrel{\rm i.\,i.\,d.}\sim}}
\newcommand{\const}{\mathrm{const}}
\usepackage[inline]{enumitem}

% вместо горизонтальной делаем косую черточку в нестрогих неравенствах
\renewcommand{\le}{\leqslant}
\renewcommand{\ge}{\geqslant}
\renewcommand{\leq}{\leqslant}
\renewcommand{\geq}{\geqslant}

\usepackage[backend=biber,bibencoding=utf8,sorting=nty,maxcitenames=4,style=numeric-verb]{biblatex}

\addbibresource{lit.bib}
\usepackage{csquotes}

\usepackage{multicol}
\usepackage{enumitem}

\renewcommand{\rmdefault}{cmss}
%\renewcommand{\ttdefault}{cmss}
\usepackage{sfmath}

\usepackage{pgfplotstable}
  
\begin{document}
	
	\maketitle
	
\section{Алгебра матриц}

\begin{itemize}
	\item \textbf{Матрица} – таблица чисел. Числа, стоящие внутри матрицы называются её \textbf{элементами}. Чтобы назвать элемент, говорят номер столбца и строки, в которых он находится. Например, в матрице $\begin{pmatrix}
	2 & 3 & 1 \\
	1 & 4 & 0
	\end{pmatrix}$ (размер: 2$\times3$), элемент, стоящий в первой строке и втором столбце – это число 3. 
	
	\item \textbf{Размер матрицы} – число строк и столбцов матрицы, записанное следующим образом: $n \times k$ ($n$ – число строк, $k$ – число столбцов).
	
	\item \textbf{Вектор-столбец} – матрица размера $n\times 1$. Пример: $\begin{pmatrix}
	1 \\
	1
	\end{pmatrix}$ (размер: 2$\times1$).
	
	\item \textbf{Вектор-строка} – матрица размера $1\times k$. Пример: $\begin{pmatrix}
	1 & 2 & 3
	\end{pmatrix}$ (размер: 1$\times3$).
	
	\item \textbf{Важно:} когда мы говорим о \textit{векторе}, мы всегда имеем в виду \textit{вектор-столбец}.
	
	\item \textbf{Транспонированная матрица} $A^T$ для матрицы $A$ – это матрица, в которой строки матрицы $A$ «поменяли местами» с её столбцами, то есть первая строка матрицы $A$ «стала» первым столбцом матрицы $A^T$, а первый столбец матрицы $A$ «стал» первой строкой матрицы $A^T$ и т.д. Пример: $A = \begin{pmatrix}
	1 & 2 \\
	3 & 4
	\end{pmatrix}$, тогда $A^T = \begin{pmatrix}
	1 & 3 \\
	2 & 4
	\end{pmatrix}$. Понятно, что $\left(A^T\right)^T = A$.
	
	\item \textbf{Квадратная матрица} – матрица, у которой число строк равно числу столбцов (то есть $n = k$). Пример: $\begin{pmatrix}
	1 & 2 \\
	3 & 4
	\end{pmatrix}$ (размер: 2$\times2$).
	
	\item \textbf{Главная диагональ} – диагональ квадратной матрицы, проведённая из левого верхнего элемента  (то есть элемента в первой строке первого столбца) в правый нижний (то есть элемента в последней строке последнего столбца). В примере выше элементы, стоящие на главной диагонали – это 1 и 4. 
	
	\item \textbf{Побочная диагональ} – диагональ квадратной матрицы, проведённая из правого верхнего элемента  (то есть элемента в первой строке последнего столбца) в левый нижний (то есть элемента в последней строке первого столбца). В примере выше элементы, стоящие на побочной диагонали – это 2 и 3. 
	
	\item \textbf{Симметричная матрица} – квадратная матрица, элементы которой симметричны относительно главной диагонали. То есть в $i$-ой строке $j$-го столбца стоит элемент, совпадающий с элементом $j$-ой строки $i$-го столбца, если $i\ne j$. Пример: $\begin{pmatrix}
	1 & 3 & 0 \\
	3 & 4 & 1 \\
	0 & 1 & 9
	\end{pmatrix}$ (размер: 3$\times3$) – симметричная матрица. На главной диагонали стоят числа 1, 4, 9. Элементы во:
	\begin{itemize}
		\item 2 строке 1 столбца и 1 строке 2 столбца совпадают (=3).
		\item 3 строке 1 столбца и 1 строке 3 столбца совпадают (=0).
		\item 3 строке 2 столбца и 2 строке 3 столбца совпадают (=1).
	\end{itemize}

	\item \textbf{Диагональная матрица} – симметричная матрица, в которой все элементы, кроме стоящих на главной диагонали, равны 0. Примеры: $\begin{pmatrix}
	1 & 0 \\
	0 & 4
	\end{pmatrix}$ (размер: 2$\times2$). $\begin{pmatrix}
	1 & 0 & 0 \\
	0 & 4 & 0 \\
	0 & 0 & 1
	\end{pmatrix}$ (размер: 3$\times3$).
	
	\item \textbf{Единичная матрица} – диагональная матрица, на главной диагонали которой стоят 1. Примеры: $\begin{pmatrix}
	1 & 0 \\
	0 & 1
	\end{pmatrix}$ (размер: 2$\times2$). $\begin{pmatrix}
	1 & 0 & 0 \\
	0 & 1 & 0 \\
	0 & 0 & 1
	\end{pmatrix}$ (размер: 3$\times3$). Мы будем обозначать единичную матрицу как $I_{n\times n}$, где $n$ – число строк и столбцов.
	
\end{itemize}

\subsection{}
Определите размер следующих векторов и матриц:
\begin{multicols}{3}
	\begin{enumerate}[label=\alph*)]
		\item $H_{n\times k}$ \\ {\small Ответ: $n\times k$.}
		\item $\begin{pmatrix}
		1 \\ 2 \\ 3
		\end{pmatrix}$ \\ {\small Ответ: $3\times 1$.}
		\item $\begin{pmatrix}
		4 & 5 & 6
		\end{pmatrix}$ \\ {\small Ответ: $1\times 3$.}
		
		\item $\begin{pmatrix}
			a & b \\
			c & d \\
			k & m
		\end{pmatrix}$ \\ {\small Ответ: $3\times 2$.}
		
		\item $\begin{pmatrix}
			1 & 1 & 1 \\
			2 & 2 & 2 \\
			3 & 3 & 3
		\end{pmatrix}$ \\ {\small Ответ: $3\times 3$.}
		
		\item $I_{n\times n}$ \\ {\small Ответ: $n\times n$.}
	\end{enumerate}
\end{multicols}

\begin{itemize}
	\item Для векторов определены стандартные операции сложения и умножения на число. Сумма двух векторов одинакового размера – это вектор, каждая координата которого равна сумме координат слагаемых. Произведение вектора на число – это произведение каждой координаты этого вектора с этим числом. В случае векторов $2\times 1$ это можно записать так:
	\[
	\begin{pmatrix}
	a_1 \\ a_2
	\end{pmatrix} + c\begin{pmatrix}
	b_1 \\ b_2
	\end{pmatrix} = \begin{pmatrix}
	a_1 + cb_1 \\ a_2 + cb_2
	\end{pmatrix}.
	\]
	Заметим, что последний вектор тоже имеет размер $2 \times 1$.
\end{itemize}


\subsection{}
Вычислите:
\begin{multicols}{2}
		\begin{enumerate}[label=\alph*)]
		\item $\begin{pmatrix}
		1 \\ 2 \\ 4
		\end{pmatrix} + \begin{pmatrix}
		4 \\ 1 \\ 2
		\end{pmatrix} = \begin{pmatrix}
		5 \\ 3 \\ 6
		\end{pmatrix}$
		
		\item $\begin{pmatrix}
		0 \\ 2 \\ 3
		\end{pmatrix} + 2\begin{pmatrix}
		3 \\ 1 \\ 8
		\end{pmatrix} = \begin{pmatrix}
		6 \\ 4 \\ 19
		\end{pmatrix}$
		
		\item $ 0_{4 \times 1} + \begin{pmatrix}
		1 \\ 1  \\ 1 \\ 1
		\end{pmatrix} = \begin{pmatrix}
		0 \\ 0  \\ 0 \\ 0
		\end{pmatrix} + \begin{pmatrix}
		1 \\ 1  \\ 1 \\ 1
		\end{pmatrix} = \begin{pmatrix}
		1 \\ 1  \\ 1 \\ 1
		\end{pmatrix}$
		
		\item $3\begin{pmatrix}
		7 \\ -1
		\end{pmatrix} - 2 \begin{pmatrix}
		1 \\ 4
		\end{pmatrix} = \begin{pmatrix}
		19 \\ -11
		\end{pmatrix}$
	\end{enumerate}
\end{multicols}

\begin{itemize}
	\item Для матриц также определены стандартные операции сложения и умножения на число. Сумма двух матриц одинакового размера – это матрица, каждый элемент которой равен сумме соответствующих элементов слагаемых (то есть стоящих в той же строке того же столбца). Произведение матрицы на число – это произведение каждого элемента этой матрицы с этим числом. В случае матрицы $2\times 2$ это можно записать так:
	\[
	\begin{pmatrix}
	a_1 & a_2 \\ a_3 & 4
	\end{pmatrix} + c\begin{pmatrix}
	b_1 & b_2 \\ b_3 & b_4
	\end{pmatrix} = \begin{pmatrix}
	a_1 + cb_1 & a_2 + cb_2 \\ a_3 + cb_3 & a_4 + cb_4 
	\end{pmatrix}.
	\]
	Заметим, что последняя матрица тоже имеет размер $2 \times 2$.
\end{itemize}

\subsection{}
Вычислите:
	\begin{enumerate}[label=\alph*)]
		\item $\begin{pmatrix}
		1 & 1 \\
		2 & 7 
		\end{pmatrix} + \begin{pmatrix}
		3 & 1 \\
		0 & 3
		\end{pmatrix} = \begin{pmatrix}
		4 & 2 \\
		2 & 10
		\end{pmatrix} $
		
		\item $\begin{pmatrix}
		4 & 4 & 1 \\
		2 & 0 & 9 \\
		1 & 1 & 0
		\end{pmatrix} - 3\begin{pmatrix}
		0 & 0 & 1 \\
		9 & 1 & 2 \\
		1 & 0 & 1
		\end{pmatrix} = \begin{pmatrix}
		4 & 4 & -2 \\
		-25 & -3 & 3 \\
		-2 & 1 & -3
		\end{pmatrix}$
		
		\item $ I + \begin{pmatrix}
		1 & 4 \\
		2 & 2
		\end{pmatrix} = \begin{pmatrix}
		2 & 4 \\
		2 & 3
		\end{pmatrix} $
		
		\item $a \begin{pmatrix}
		3 & 0 & 0 \\
		0 & 1 & 1 \\
		0 & 2 & 4 \\
		\end{pmatrix} + kI = \begin{pmatrix}
		3a + k & 0 & 0 \\
		0 & a + k & a \\
		0 & 2a & 4a + k \\
		\end{pmatrix} $
	\end{enumerate}

\begin{itemize}
	\item \textbf{Матричное произведение} – операция, обладающая следующими свойствами:
	\begin{enumerate}
		\item Матрицы $A$ и $B$ можно матрично перемножить, только если число столбцов матрицы $A$ равно числу строк матрицы $B$.
		\item Результат матричного произведения – матрица $C$, размер которой равен: (число строк матрицы $A$) $\times$ (число столбцов матрицы $B$).
		\[
		A_{n\times k} \times B_{k \times m} = C_{n \times m}.
		\]
		\item Каждый элемент матрицы $C$ равен сумме поэлементных произведений элементов соответствующей строки матрицы $A$ и столбца матрицы $B$. То есть: 
		\[
		\begin{pmatrix}
		a_1 & a_2 \\
		a_3 & a_4
		\end{pmatrix} \times \begin{pmatrix}
		b_1 & b_2 \\
		b_3 & b_4
		\end{pmatrix} = \begin{pmatrix}
		a_1b_1 + a_2b_3 & a_1b_2 + a_2b_4 \\
		a_3b_1 + a_4b_3 & a_3b_2 + a_4b_4
		\end{pmatrix}.
		\]
	\end{enumerate}
\end{itemize}

\subsection{}
Вычислите матричное произведение, если это возможно:
	\begin{enumerate}[label=\alph*)]
		
		\item $I_{n \times n} \times H_{m\times k} = \begin{cases}
		H_{n\times k}, \text{ если } n = m \\
		\emptyset, \text{ иначе}
		\end{cases}$
		
		\item $\begin{pmatrix}
			1 & 3 \\
			2 & 1 
		\end{pmatrix} \times \begin{pmatrix}
		2 \\ 4
		\end{pmatrix}  = \begin{pmatrix}
		14 \\ 8
		\end{pmatrix}$
		
		\item $\begin{pmatrix}
		7 & 4 & 1 \\
		1 & 2 & 2
		\end{pmatrix} \times \begin{pmatrix}
		3 & 2 \\
		1 & 1 \\
		0 & 4
		\end{pmatrix} = \begin{pmatrix}
		25  & 22 \\
		5 & 12
		\end{pmatrix}$
		
		\item $\begin{pmatrix}
		1 & 1 \\
		2 & 3
		\end{pmatrix} \times \begin{pmatrix}
		1 & 3
		\end{pmatrix}$ – невозможно, т.к. размеры $2\times2$ и $1\times2$
		
		\item $7\begin{pmatrix}
		3 & 3 \\
		1 & 2
		\end{pmatrix}\times I \times \begin{pmatrix}
		7 & 1 \\
		0 & 2
		\end{pmatrix} = 7\begin{pmatrix}
		3 & 3 \\
		1 & 2
		\end{pmatrix} \times \begin{pmatrix}
		7 & 1 \\
		0 & 2
		\end{pmatrix} = 7 \begin{pmatrix}
		21 & 9 \\
		7 & 5
		\end{pmatrix} = \begin{pmatrix}
		147 & 63 \\
		49 & 35
		\end{pmatrix}$
		
		\item $\begin{pmatrix}
		3 & 0 \\
		0 & 3
		\end{pmatrix}\times \begin{pmatrix}
		1 & 4 \\
		2 & 1
		\end{pmatrix} =\begin{pmatrix}
		3 & 12 \\
		6 & 3
		\end{pmatrix} $
	\end{enumerate}


\section{Определитель}

\textbf{Определитель} – некоторое число, характеризующее свойства матрицы. Если $A$ – некоторая матрица, то определитель $A$ обозначается следующим образом:
\[
\det(A) \text{ или } |A|
\]

Свойства определителя:
\begin{enumerate}
	\item Определитель можно рассчитать \textbf{только для квадратных} матриц.
	\item Общий множитель элементов некоторой строки можно выносить за знак определителя:
	\[
	\det(cA_{n\times n}) = c^n\det(A), c \in \R
	\]
	Пример: $\begin{vmatrix}
	2 & 2 \\
	3 & 1
	\end{vmatrix} = 2 \begin{vmatrix}
	1 & 1 \\
	3 & 1
	\end{vmatrix}$.
	
	Пример: $\begin{vmatrix}
	2 & 2 \\
	6 & 2
	\end{vmatrix} = 2 \times 2\begin{vmatrix}
	1 & 1 \\
	3 & 1
	\end{vmatrix}$.
	\item Если матрица имеет две одинаковые строки (столбца) или нулевую строку (столбец), то её определитель равен 0.
	\item $\det(I) = 1$.
	\item Определитель диагональной матрицы равен произведению элементов главной диагонали. 
	\item В случае матрицы размера $2\times2$ определитель можно найти по следующей формуле:
	\[
	\begin{vmatrix}
	a_1 & a_2 \\
	a_3 & a_4
	\end{vmatrix} =a_1a_4 - a_3a_2
	\]
\end{enumerate}

\subsection{}
Найдите определитель следующих матриц, если это возможно:
		\begin{enumerate}[label=\alph*)]
			\item $\begin{vmatrix}
			2 & 4 \\
			6 & 1
			\end{vmatrix} = 2 - 24 = -22$
			
			\item $\begin{vmatrix}
			9 & 2 \\
			0 & 1
			\end{vmatrix} = 9$
			
			\item $\det\begin{pmatrix}
			1 \\
			1
			\end{pmatrix}$ – невозможно, так как не квадратная.
			
			\item $\det(1) = 1$
			
			\item $\det(I) = 1$
			
			\item $\begin{vmatrix}
			7 & 0 \\
			1 & 2
			\end{vmatrix} = 14$
			
			\item $\begin{vmatrix}
			a & b \\
			c & d
			\end{vmatrix} = ad - bc$
			
			\item $\begin{vmatrix}
			1 & 1 \\
			2 & 2
			\end{vmatrix} = 2\begin{vmatrix}
			1 & 1 \\
			1 & 1
			\end{vmatrix} = 0$ (две одинаковые строки)
		\end{enumerate}

\textbf{Обратной матрицей} $A^{-1}$ для матрицы $A$ называется матрица, обладающая следующим свойством:
\[
A^{-1} \times A = A \times A^{-1} = I
\]
Свойства обратной матрицы:
\begin{enumerate}
	\item Обратная матрица существует только для \textbf{квадратных матриц с ненулевым определителем}.
	\item В случае матрицы размера $2\times2$ обратную матрицу можно найти по следующей формуле:
	\[
	\text{если }A = \begin{pmatrix}
	a_1 & a_2 \\
	a_3 & a_4
	\end{pmatrix}, \text{то } A^{-1} = \dfrac{1}{\det(A)}\begin{pmatrix}
	a_4 & -a_2 \\
	-a_3 & a_1
	\end{pmatrix}
	\]
	\item Для квадратной диагональной матрицы обратная матрица получается путём замены элементов, стоящих на главной диагонали, на обратные к ним (то есть $a_{ii}$ заменяется на $1/a_{ii}$). 
\end{enumerate}


\subsection{}

Найдите матрицу, обратную данной, если это возможно:
	\begin{enumerate}[label=\alph*)]
		\item $A = \begin{pmatrix}
		1 & 2 \\
		3 & 4
		\end{pmatrix} \Rightarrow A^{-1} = -\dfrac{1}{2}\begin{pmatrix}
		4 & -2 \\
		-3 & 1
		\end{pmatrix} $
		\item $\begin{pmatrix}
		1 & 1 \\
		2 & 2
		\end{pmatrix}$ – невозможно, так как определитель исходной матрицы равен 0.
		\item $A = \begin{pmatrix}
		4
		\end{pmatrix} \Rightarrow A^{-1} = \dfrac{1}{4}$
		\item $A = \begin{pmatrix}
		1 & 0 & 0 \\
		0 & 3 & 0 \\
		0 & 0 & 4 
		\end{pmatrix} \Rightarrow A^{-1} = \begin{pmatrix}
		1 & 0 & 0 \\
		0 & 1/3 & 0 \\
		0 & 0 & 1/4 
		\end{pmatrix}$
		\item $A = \begin{pmatrix}
			7 & 1 \\
			0 & 2
		\end{pmatrix} \Rightarrow A^{-1} = \dfrac{1}{14}\begin{pmatrix}
		2 & -1 \\
		0 & 7
		\end{pmatrix}$
		\item $\begin{pmatrix}
			6 & 1 & 2 \\
			3 & 1 & 0
		\end{pmatrix}$ – невозможно, так как не квадратная.
	\end{enumerate}

Векторы $b_1, b_2 \ldots b_n$ называют \textbf{линейно зависимыми} (ЛЗ), если существую такие числа $a_1, a_2 \ldots a_n \in \R$, из которых хотя бы одно не равно 0, что верно следующее равенство:
\[
a_1b_1 + a_2b_2 + \ldots a_nb_n = 0
\]
Если же это равенство верно только при всех $a_i = 0$, то векторы $b_1, b_2 \ldots b_n$ называют \textbf{линейно независимыми} (ЛНЗ).

Пример: векторы $\begin{pmatrix}
1 \\
2
\end{pmatrix}$ и $\begin{pmatrix}
0 \\
3
\end{pmatrix}$ – линейно независимы, так как система уравнений:
\[
a_1\begin{pmatrix}
1 \\
2
\end{pmatrix} + a_2\begin{pmatrix}
0 \\
3
\end{pmatrix} = 0 \Rightarrow \begin{pmatrix}
a_1 \\
2a_1 + 3a_2
\end{pmatrix} = \begin{pmatrix}
0 \\
0
\end{pmatrix}
\]
– имеет единственное решение при $a_1 = 0$ и $a_2 = 0$.

Пример: векторы $b_1 = \begin{pmatrix}
1 \\
1
\end{pmatrix}$ и $b_2 = \begin{pmatrix}
2 \\
2
\end{pmatrix}$ – линейно зависимы, так как
\[
a_1b_1 + a_2b_2 = 0
\]
при, например, $a_1 = -2$, $a_2 = 1$ (для доказательства линейной зависимости достаточно привести хотя бы одно $a_i \ne 0$).

\textbf{Правило:} определитель системы линейно зависимых векторов равен 0.

\subsection{}
Определите, являются ли векторы линейно зависимыми. Если дана матрица, определите, есть ли в ней линейно зависимые векторы.
	\begin{enumerate}[label=\alph*)]
		\item $\begin{pmatrix}
		1 \\ 0
		\end{pmatrix}$ и $\begin{pmatrix}
		0 \\ 1
		\end{pmatrix}$ – ЛНЗ
		
		\item $\begin{pmatrix}
		1 \\ 1
		\end{pmatrix}$ и $\begin{pmatrix}
		3 \\ 3
		\end{pmatrix}$ - ЛЗ, $a_1 = -3$, $a_2 = 1$
		
		\item $\begin{pmatrix}
		3 \\ 2 \\ 3
		\end{pmatrix}$ и $\begin{pmatrix}
		27 \\ 14 \\ 27
		\end{pmatrix}$ – ЛНЗ
		
		\item $\begin{pmatrix}
		3 \\ 2 \\ 3
		\end{pmatrix}$ и $\begin{pmatrix}
		9 \\ 6 \\ 9
		\end{pmatrix}$ – ЛЗ, $a_1 = -3$, $a_2 = 1$
		
		\item $\begin{pmatrix}
		3 & 1 & 2 \\
		3 & 4 & 5 \\
		18 & 6 & 12
		\end{pmatrix}$ – есть ЛЗ, т.к. третья строка – это первая строка, умноженная на 3.
		
		\item $\begin{pmatrix}
		1 & 2 \\
		2 & 1
		\end{pmatrix}$ – нет ЛЗ (определитель не равен 0)
		
		\item $\begin{pmatrix}
		1 & 1 & 0 \\
		0 & 3 & 0 \\
		1 & 12 & 0
		\end{pmatrix}$ – есть ЛЗ (определитель равен 0, так как есть нулевой столбец)
		
		\item $\begin{pmatrix}
		1 & 1 & 3 \\
		4 & 2 & 6
		\end{pmatrix}$ – есть ЛЗ (третий столбец равен второму столбцу, умноженному на 3)
	\end{enumerate}

	
\section{Векторы в пространстве}

	\textbf{Скалярное произведение векторов} – операция, выполняемая по следующему правилу:
	\[
	\text{Для векторов $a_{n\times1}$ и $b_{n\times1}$: } \langle a, b \rangle = a_1b_1 + a_2b_2 + \ldots a_nb_n
	\] 
	
	\subsection{}
	Вычислите скалярное произведение векторов:
		\begin{enumerate}[label=\alph*)]
			
			\item $\langle\begin{pmatrix}
			1 \\ 1
			\end{pmatrix}, \begin{pmatrix}
			2 \\ 4
			\end{pmatrix}\rangle = 2 + 4 = 6$
			
			\item $\langle\begin{pmatrix}
			a \\ m \\ d
			\end{pmatrix}, \begin{pmatrix}
			c \\ k \\ p
			\end{pmatrix}\rangle = ac + mk + dp$
			
			\item $\langle a_{n\times1}, b_{n\times1}\rangle = a_1b_1 + a_2b_2 + \ldots a_nb_n$
		\end{enumerate}
	
	\textbf{Норма (длина) вектора} вычисляется по следующей формуле:
	\[
	||a_{n\times1}|| = \sqrt{a_1^2 + a_2^2 + \ldots a_n^2} = \sqrt{\langle a,a\rangle}
	\]

	\subsection{}
	Вычислите:
	\begin{multicols}{2}
		\begin{enumerate}[label=\alph*)]
			\item $a = \begin{pmatrix}
			1 \\
			2
			\end{pmatrix}, ||a|| = \sqrt{5}$
			\item $b = \begin{pmatrix}
			5 \\
			2 \\ 
			1
			\end{pmatrix}, ||b||^2 = 25 + 4 + 1 = 30$
			\item $c = \begin{pmatrix}
			c_1 \\
			c_2 \\
			\vdots \\
			c_n
			\end{pmatrix}, ||c|| = \sqrt{c_1^2 + c_2^2 + \ldots c_n^2}$
			\item $d = \begin{pmatrix}
			1 \\
			0 \\
			0 \\
			\end{pmatrix}, ||d||^2 = 1$
		\end{enumerate}
	\end{multicols}
	
	Косинус угла между векторами $x$ и $y$ вычисляется по следующей формуле:
	\[
	\cos(\angle x, y) = \dfrac{\langle x, y\rangle}{||x||\times||y||}
	\] 
	
	\textbf{Ортогональные (перпендикулярные)} векторы – векторы, косинус угла между которыми равен 0.
	
	\textbf{Важное следствие:} векторы ортогональны, когда их скалярное произведение равно 0. 
	
	\subsection{}	
	Найдите косинус угла между векторами (обозначим этот угол как $\alpha$). Определите, являются ли векторы ортогональными. 
	
		\begin{enumerate}[label=\alph*)]
			\item $\begin{pmatrix}
			1 \\
			2 \\
			4
			\end{pmatrix}$, $\begin{pmatrix}
			2 \\
			1 \\
			0
			\end{pmatrix}$ $\Rightarrow \cos(\alpha) = \dfrac{4}{\sqrt{105}}$, не ортогональны
			\item $\begin{pmatrix}
			1 \\
			2
			\end{pmatrix}$, $\begin{pmatrix}
			-1 \\
			0
			\end{pmatrix}$ $\Rightarrow \cos(\alpha) = \dfrac{-1}{\sqrt{5}}$, не ортогональны
			\item $\begin{pmatrix}
			3 \\
			4 \\
			1
			\end{pmatrix}$, $\begin{pmatrix}
			1 \\
			-1 \\
			1
			\end{pmatrix}$ $\Rightarrow \cos(\alpha) = 0$, ортогональны
			\item $\begin{pmatrix}
			1 \\
			2 \\
			1
			\end{pmatrix}$, $\begin{pmatrix}
			-1 \\
			1 \\
			0
			\end{pmatrix}$ $\Rightarrow \cos(\alpha) = \dfrac{1}{\sqrt{12}}$, не ортогональны
			\item $\begin{pmatrix}
			-1 \\
			-2 \\
			6
			\end{pmatrix}$, $\begin{pmatrix}
			4 \\
			-2 \\
			0
			\end{pmatrix}$ $\Rightarrow \cos(\alpha) = 0$, ортогональны
			\item $\begin{pmatrix}
			0 \\
			-2 \\
			4 \\
			1
			\end{pmatrix}$, $\begin{pmatrix}
			12 \\
			-1 \\
			-1 \\
			2
			\end{pmatrix}$ $\Rightarrow \cos(\alpha) = 0$, ортогональны
		\end{enumerate}

	\subsection{}
	Изобразите следующие векторы и системы векторов. Определите, содержит ли система векторов линейно зависимые векторы. 
		\begin{multicols}{2}
		\begin{enumerate}[label=\alph*)]
			\item $\begin{pmatrix}
			1 \\
			2
			\end{pmatrix}$ – в системе координат XY
			\item $\begin{pmatrix}
			1 \\
			0
			\end{pmatrix}$, 
			$\begin{pmatrix}
			0 \\
			1
			\end{pmatrix}$ – в системе координат XY, ЛНЗ
			\item$\begin{pmatrix}
			5 & 1 & 0 \\
			4 & 0 & 1
			\end{pmatrix}$ – в системе координат XY, ЛЗ
			\item $\begin{pmatrix}
			4 & 2 \\
			1 & 0.5 \\
			2 & 0
			\end{pmatrix}$ – в системе координат XYZ, ЛНЗ
			\item $\begin{pmatrix}
			1 & 2 & 0 \\
			1 & 2 & 1 \\
			1 & 2 & 0 \\
			\end{pmatrix}$ – в системе координат XYZ, ЛЗ
			\item $y_n$, $x_n$, $\langle y, x \rangle = 0$ – любые два ортогональных вектора.
			\item $y_k$, $x_k$, $z_k$, $||x||^2 + ||y||^2 = ||z||^2$ – любые три вектора, образующие прямоугольный треугольник с гипотенузой $z$.
			\item $y_{100}$, $p_{100}$, $m_{100}$, $m_{100} = y_{100} + 2p_{100}$ – любые три вектора из 100 элементов, один из которых равен сумме второго и удвоенного третьего.
		\end{enumerate}
	\end{multicols}
	
	\textbf{Линейное пространство (неформально)} – вектор или система векторов, удовлетворяющие некоторым аксиомам. В частности, для них должны быть определены операции сложения и умножения на число.
	
	\textbf{Ортогональная проекция} – изображение вектора, лежащего в одном линейном пространстве, в другом линейном пространстве так, чтобы расстояние между исходным вектором и его проекцией было минимальным. Для вектора $Y$ ортогональную проекцию будем обозначать как  $\hat{Y}$.
	
	\subsection{}
	Изобразите проекцию вектора $Y$ на указанное пространство. В пункте a) рассчитайте координаты проекции.
		\begin{enumerate}[label=\alph*)]
		\item $Y = \begin{pmatrix}
		3 \\
		1 \\
		2 \\
		\end{pmatrix}$ на $a = \begin{pmatrix}
		1 \\
		1 \\
		1 \\
		\end{pmatrix}$
		
		\begin{center}
			
			
			\tikzset{every picture/.style={line width=0.75pt}} %set default line width to 0.75pt        
			
			\begin{tikzpicture}[x=0.75pt,y=0.75pt,yscale=-1,xscale=1]
			%uncomment if require: \path (0,391); %set diagram left start at 0, and has height of 391
			
			%Straight Lines [id:da2741950629482076] 
			\draw [color={rgb, 255:red, 74; green, 144; blue, 226 }  ,draw opacity=1 ][line width=1.5]    (18,162) -- (269.5,160.02) ;
			\draw [shift={(272.5,160)}, rotate = 539.55] [color={rgb, 255:red, 74; green, 144; blue, 226 }  ,draw opacity=1 ][line width=1.5]    (14.21,-4.28) .. controls (9.04,-1.82) and (4.3,-0.39) .. (0,0) .. controls (4.3,0.39) and (9.04,1.82) .. (14.21,4.28)   ;
			
			%Straight Lines [id:da5774161161515429] 
			\draw    (18,162) -- (181.91,36.22) ;
			\draw [shift={(183.5,35)}, rotate = 502.5] [color={rgb, 255:red, 0; green, 0; blue, 0 }  ][line width=0.75]    (10.93,-3.29) .. controls (6.95,-1.4) and (3.31,-0.3) .. (0,0) .. controls (3.31,0.3) and (6.95,1.4) .. (10.93,3.29)   ;
			
			%Straight Lines [id:da23826234440520855] 
			\draw [color={rgb, 255:red, 208; green, 2; blue, 27 }  ,draw opacity=1 ]   (18,162) -- (182.5,161.01) ;
			\draw [shift={(184.5,161)}, rotate = 539.6600000000001] [color={rgb, 255:red, 208; green, 2; blue, 27 }  ,draw opacity=1 ][line width=0.75]    (10.93,-3.29) .. controls (6.95,-1.4) and (3.31,-0.3) .. (0,0) .. controls (3.31,0.3) and (6.95,1.4) .. (10.93,3.29)   ;
			
			%Straight Lines [id:da055136705529606145] 
			\draw  [dash pattern={on 4.5pt off 4.5pt}]  (183.5,35) -- (184.5,161) ;
			
			
			
			% Text Node
			\draw (161,27) node   {$Y$};
			% Text Node
			\draw (129,180) node [color={rgb, 255:red, 208; green, 2; blue, 27 }  ,opacity=1 ]  {$\hat{Y}$};
			% Text Node
			\draw (286,159) node [color={rgb, 255:red, 74; green, 144; blue, 226 }  ,opacity=1 ]  {$a$};
			\end{tikzpicture}
		\end{center}
	
		По определению, ортогональная проекция строится так, чтобы расстояние от исходного вектора до его проекции было минимальным. По рисунку видно, что это расстояние – это длина вектора ($Y - \hat{Y})$. Заметим, что так как мы проецируем вектор $Y$ на вектор $a$, то $\hat{Y}$ будет лежать на $a$, то есть $a$ и $\hat{Y}$ – коллинеарны. Это означает, что $\hat{Y} = pa$, где $p$ – некоторое число, не равное 0. Тогда нужно минимизировать:
		\begin{align*}
			||Y - \hat{Y}|| = ||Y - pa|| = \sqrt{(3 - p)^2 + (1 - p)^2 + (2 - p)^2}.
		\end{align*}
		Понятно, что если мы минимизируем корень, то мы минимизируем подкоренное выражение. Тогда:
		\[
		(3 - p)^2 + (1 - p)^2 + (2 - p)^2 = 3p^2 -12p + 14
		\]
		Парабола с ветвями вверх относительно $p$ $\Rightarrow$ минимум достигается в вершине при $p = 2$.
		
		Тогда координаты проекции: 
		\[
		\hat{Y} = pa = \begin{pmatrix}
		2 \\
		2 \\
		2
		\end{pmatrix}
		\] 
		
		Заметим, что мы так же могли бы решить другую задачу:
		\[
		\langle Y - \hat{Y}, a \rangle = 0,
		\]
		так как, как известно, наименьшее расстояние между точкой (концом вектора $Y$) и прямой $a$ проходит по перпендикуляру от точки до прямой. То есть чтобы расстояние между $Y$ и $\hat{Y}$ было минимальным, $Y - \hat{Y}$ и $a$ должны быть перпендикулярны.
		
		Тогда подставляем:
		\[
		\langle \begin{pmatrix}
		3 - p \\
		1 - p \\
		2 - p
		\end{pmatrix}, \begin{pmatrix}
		1 \\ 1 \\ 1
		\end{pmatrix}  \rangle = 0,
		\]
		откуда $3p = 3 + 1 + 2 \Rightarrow p = \dfrac{3 + 1 + 2}{3} = 2$, как и получилось до этого.
		
		Заметим, что $p$ – это среднее арифметическое из координат вектора $Y$ (среднее по координатам вектора обозначается как $\bar{Y} = \dfrac{Y_1 + \ldots Y_n}{n}$), то есть $\hat{Y} = \bar{Y} \begin{pmatrix}
		1 \\ 1 \\ 1
		\end{pmatrix}$. Это свойство выполнено для \textbf{любого} вектора $Y$: при проецировании вектора $Y$ на вектор из единиц проекцией будет являться вектор из средних по элементам вектора $Y$.
		
		\item $Y = \begin{pmatrix}
		y_1 \\
		\vdots \\
		y_n
		\end{pmatrix}$ на $X = \begin{pmatrix}
		x_{11} & x_{12} \\
		\vdots & \vdots \\
		x_{n1} & x_{n2}
		\end{pmatrix}$
		
		\begin{center}
			\tikzset{every picture/.style={line width=0.75pt}} %set default line width to 0.75pt        
			
			\begin{tikzpicture}[x=0.75pt,y=0.75pt,yscale=-1,xscale=1]
			%uncomment if require: \path (0,391); %set diagram left start at 0, and has height of 391
			
			%Straight Lines [id:da5774161161515429] 
			\draw    (18,162) -- (181.91,36.22) ;
			\draw [shift={(183.5,35)}, rotate = 502.5] [color={rgb, 255:red, 0; green, 0; blue, 0 }  ][line width=0.75]    (10.93,-3.29) .. controls (6.95,-1.4) and (3.31,-0.3) .. (0,0) .. controls (3.31,0.3) and (6.95,1.4) .. (10.93,3.29)   ;
			
			%Straight Lines [id:da8344391108235314] 
			\draw [color={rgb, 255:red, 74; green, 144; blue, 226 }  ,draw opacity=1 ]   (18,162) -- (201.54,122.42) ;
			\draw [shift={(203.5,122)}, rotate = 527.8299999999999] [color={rgb, 255:red, 74; green, 144; blue, 226 }  ,draw opacity=1 ][line width=0.75]    (10.93,-3.29) .. controls (6.95,-1.4) and (3.31,-0.3) .. (0,0) .. controls (3.31,0.3) and (6.95,1.4) .. (10.93,3.29)   ;
			
			%Straight Lines [id:da8525361107121421] 
			\draw [color={rgb, 255:red, 74; green, 144; blue, 226 }  ,draw opacity=1 ][fill={rgb, 255:red, 245; green, 166; blue, 35 }  ,fill opacity=1 ]   (18,162) -- (206.56,208.52) ;
			\draw [shift={(208.5,209)}, rotate = 193.86] [color={rgb, 255:red, 74; green, 144; blue, 226 }  ,draw opacity=1 ][line width=0.75]    (10.93,-3.29) .. controls (6.95,-1.4) and (3.31,-0.3) .. (0,0) .. controls (3.31,0.3) and (6.95,1.4) .. (10.93,3.29)   ;
			
			%Curve Lines [id:da10289916094042029] 
			\draw [color={rgb, 255:red, 74; green, 144; blue, 226 }  ,draw opacity=1 ]   (203.5,122) .. controls (421.5,93) and (506.5,268) .. (208.5,209) ;
			
			
			%Straight Lines [id:da23826234440520855] 
			\draw [color={rgb, 255:red, 208; green, 2; blue, 27 }  ,draw opacity=1 ]   (18,162) -- (181.5,170.89) ;
			\draw [shift={(183.5,171)}, rotate = 183.11] [color={rgb, 255:red, 208; green, 2; blue, 27 }  ,draw opacity=1 ][line width=0.75]    (10.93,-3.29) .. controls (6.95,-1.4) and (3.31,-0.3) .. (0,0) .. controls (3.31,0.3) and (6.95,1.4) .. (10.93,3.29)   ;
			
			%Straight Lines [id:da055136705529606145] 
			\draw  [dash pattern={on 4.5pt off 4.5pt}]  (183.5,35) -- (183.5,171) ;
			
			
			
			% Text Node
			\draw (161,27) node   {$Y$};
			% Text Node
			\draw (154,116) node [color={rgb, 255:red, 74; green, 144; blue, 226 }  ,opacity=1 ]  {$X_{1}$};
			% Text Node
			\draw (177,217) node [color={rgb, 255:red, 74; green, 144; blue, 226 }  ,opacity=1 ]  {$X_{2}$};
			% Text Node
			\draw (201,172) node [color={rgb, 255:red, 208; green, 2; blue, 27 }  ,opacity=1 ]  {$\hat{Y}$};
			
			
			\end{tikzpicture}
			
		\end{center}
		
		\item $Y = \begin{pmatrix}
		y_1 \\
		\vdots \\
		y_n
		\end{pmatrix}$ на $X_{n \times k}$
		
		\begin{center}
			
			
			\tikzset{every picture/.style={line width=0.75pt}} %set default line width to 0.75pt        
			
			\begin{tikzpicture}[x=0.75pt,y=0.75pt,yscale=-1,xscale=1]
			%uncomment if require: \path (0,391); %set diagram left start at 0, and has height of 391
			
			%Straight Lines [id:da5774161161515429] 
			\draw    (18,162) -- (181.91,36.22) ;
			\draw [shift={(183.5,35)}, rotate = 502.5] [color={rgb, 255:red, 0; green, 0; blue, 0 }  ][line width=0.75]    (10.93,-3.29) .. controls (6.95,-1.4) and (3.31,-0.3) .. (0,0) .. controls (3.31,0.3) and (6.95,1.4) .. (10.93,3.29)   ;
			
			%Straight Lines [id:da8344391108235314] 
			\draw [color={rgb, 255:red, 74; green, 144; blue, 226 }  ,draw opacity=1 ]   (18,162) -- (201.54,122.42) ;
			\draw [shift={(203.5,122)}, rotate = 527.8299999999999] [color={rgb, 255:red, 74; green, 144; blue, 226 }  ,draw opacity=1 ][line width=0.75]    (10.93,-3.29) .. controls (6.95,-1.4) and (3.31,-0.3) .. (0,0) .. controls (3.31,0.3) and (6.95,1.4) .. (10.93,3.29)   ;
			
			%Straight Lines [id:da8525361107121421] 
			\draw [color={rgb, 255:red, 74; green, 144; blue, 226 }  ,draw opacity=1 ][fill={rgb, 255:red, 245; green, 166; blue, 35 }  ,fill opacity=1 ]   (18,162) -- (206.56,208.52) ;
			\draw [shift={(208.5,209)}, rotate = 193.86] [color={rgb, 255:red, 74; green, 144; blue, 226 }  ,draw opacity=1 ][line width=0.75]    (10.93,-3.29) .. controls (6.95,-1.4) and (3.31,-0.3) .. (0,0) .. controls (3.31,0.3) and (6.95,1.4) .. (10.93,3.29)   ;
			
			%Curve Lines [id:da10289916094042029] 
			\draw [color={rgb, 255:red, 74; green, 144; blue, 226 }  ,draw opacity=1 ]   (203.5,122) .. controls (421.5,93) and (506.5,268) .. (208.5,209) ;
			
			
			%Straight Lines [id:da23826234440520855] 
			\draw [color={rgb, 255:red, 208; green, 2; blue, 27 }  ,draw opacity=1 ]   (18,162) -- (181.5,170.89) ;
			\draw [shift={(183.5,171)}, rotate = 183.11] [color={rgb, 255:red, 208; green, 2; blue, 27 }  ,draw opacity=1 ][line width=0.75]    (10.93,-3.29) .. controls (6.95,-1.4) and (3.31,-0.3) .. (0,0) .. controls (3.31,0.3) and (6.95,1.4) .. (10.93,3.29)   ;
			
			%Straight Lines [id:da055136705529606145] 
			\draw  [dash pattern={on 4.5pt off 4.5pt}]  (183.5,35) -- (183.5,171) ;
			
			
			%Straight Lines [id:da800593077517729] 
			\draw [color={rgb, 255:red, 74; green, 144; blue, 226 }  ,draw opacity=1 ]   (23,160) -- (209.51,142.19) ;
			\draw [shift={(211.5,142)}, rotate = 534.55] [color={rgb, 255:red, 74; green, 144; blue, 226 }  ,draw opacity=1 ][line width=0.75]    (10.93,-3.29) .. controls (6.95,-1.4) and (3.31,-0.3) .. (0,0) .. controls (3.31,0.3) and (6.95,1.4) .. (10.93,3.29)   ;
			
			%Straight Lines [id:da6247981887012621] 
			\draw [color={rgb, 255:red, 74; green, 144; blue, 226 }  ,draw opacity=1 ]   (23,160) -- (216.5,157.03) ;
			\draw [shift={(218.5,157)}, rotate = 539.12] [color={rgb, 255:red, 74; green, 144; blue, 226 }  ,draw opacity=1 ][line width=0.75]    (10.93,-3.29) .. controls (6.95,-1.4) and (3.31,-0.3) .. (0,0) .. controls (3.31,0.3) and (6.95,1.4) .. (10.93,3.29)   ;
			
			%Shape: Arc [id:dp7982623288532297] 
			\draw  [draw opacity=0] (217.57,200.02) .. controls (219.29,199.69) and (221.02,199.11) .. (222.68,198.24) .. controls (229.91,194.46) and (233.53,186.65) .. (231.74,179.69) -- (215.5,184.5) -- cycle ; \draw  [color={rgb, 255:red, 74; green, 144; blue, 226 }  ,draw opacity=1 ] (217.57,200.02) .. controls (219.29,199.69) and (221.02,199.11) .. (222.68,198.24) .. controls (229.91,194.46) and (233.53,186.65) .. (231.74,179.69) ;
			
			% Text Node
			\draw (161,27) node   {$Y$};
			% Text Node
			\draw (154,116) node [color={rgb, 255:red, 74; green, 144; blue, 226 }  ,opacity=1 ]  {$X_{1}$};
			% Text Node
			\draw (177,217) node [color={rgb, 255:red, 74; green, 144; blue, 226 }  ,opacity=1 ]  {$X_{k}$};
			% Text Node
			\draw (201,172) node [color={rgb, 255:red, 208; green, 2; blue, 27 }  ,opacity=1 ]  {$\hat{Y}$};
			% Text Node
			\draw (227,136) node [color={rgb, 255:red, 74; green, 144; blue, 226 }  ,opacity=1 ]  {$X_{2}$};
			% Text Node
			\draw (237,160) node [color={rgb, 255:red, 74; green, 144; blue, 226 }  ,opacity=1 ]  {$X_{3}$};
			
			
			\end{tikzpicture}
		\end{center}
		
		\end{enumerate}
	
\end{document}