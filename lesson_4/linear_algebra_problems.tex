\documentclass[11pt, a4paper]{extarticle}
\setlength{\parskip}{0.2em}
%%% Работа с русским языком
\usepackage{cmap}					% поиск в PDF
\usepackage{mathtext} 				% русские буквы в формулах
\usepackage[T2A]{fontenc}			% кодировка
\usepackage[utf8]{inputenc}			% кодировка исходного текста
\usepackage[english,russian]{babel}	% локализация и переносы
\usepackage{mathtools}   % loads »amsmath«
\usepackage{graphicx}
\usepackage{caption}
\usepackage{physics}
\usepackage{subcaption}
\usepackage{tikz}
\usepackage{multicol}
\usepackage{listings}
\usepackage{tabularx}

\usepackage{hyperref}
\hypersetup{				% Гиперссылки
    unicode=true,           % русские буквы в раздела PDF
    pdftitle={Заголовок},   % Заголовок
    pdfauthor={Автор},      % Автор
    pdfsubject={Тема},      % Тема
    pdfcreator={Создатель}, % Создатель
    pdfproducer={Производитель}, % Производитель
    pdfkeywords={keyword1} {key2} {key3}, % Ключевые слова
    colorlinks=true,       	% false: ссылки в рамках; true: цветные ссылки
    linkcolor=black,          % внутренние ссылки
    citecolor=black,        % на библиографию
    filecolor=cyan,      % на файлы
    urlcolor=gray           % на URL
}

\usepackage{setspace} % Интерлиньяж
%\onehalfspacing % Интерлиньяж 1.5
%\doublespacing % Интерлиньяж 2
\singlespacing % Интерлиньяж 1

%%% Дополнительная работа с математикой
\usepackage{amsmath,amsfonts,amssymb,amsthm,mathtools} % AMS
\usepackage{icomma} % "Умная" запятая: $0,2$ --- число, $0, 2$ --- перечисление

%% Шрифты
\usepackage{euscript}	 % Шрифт Евклид
\usepackage{mathrsfs} % Красивый матшрифт
\usepackage{float}

\title{Элементы Линейной Алгебры}

\author{Факультатив «Введение в анализ данных и машинное обучение на Python»}
\date{\today}

\usepackage{geometry}
\geometry{
	a4paper,
	left=20mm,
	top=20mm,
	right=20mm
}
%\setlength{\parindent}{0cm}

\DeclareMathOperator{\Lin}{\mathrm{Lin}}
\DeclareMathOperator{\Linp}{\Lin^{\perp}}
\DeclareMathOperator*\plim{plim}
%\DeclareMathOperator{\grad}{grad}
\DeclareMathOperator{\card}{card}
\DeclareMathOperator{\sgn}{sign}
\DeclareMathOperator{\sign}{sign}

\DeclareMathOperator*{\argmin}{arg\,min}
\DeclareMathOperator*{\argmax}{arg\,max}
\DeclareMathOperator*{\amn}{arg\,min}
\DeclareMathOperator*{\amx}{arg\,max}
\DeclareMathOperator{\cov}{Cov}
\DeclareMathOperator{\Var}{Var}
\DeclareMathOperator{\Cov}{Cov}
\DeclareMathOperator{\Corr}{Corr}
\DeclareMathOperator{\pCorr}{pCorr}
\DeclareMathOperator{\E}{\mathbb{E}}
\let\P\relax
\DeclareMathOperator{\P}{\mathbb{P}}



\newcommand{\cN}{\mathcal{N}}
\newcommand{\cU}{\mathcal{U}}
\newcommand{\cBinom}{\mathcal{Binom}}
\newcommand{\cPois}{\mathcal{Pois}}
\newcommand{\cBeta}{\mathcal{Beta}}
\newcommand{\cGamma}{\mathcal{Gamma}}

\def \R{\mathbb{R}}
\def \N{\mathbb{N}}
\def \Z{\mathbb{Z}}

\usepackage{multirow}
\usepackage{array}


\newcommand{\dx}[1]{\,\mathrm{d}#1} % для интеграла: маленький отступ и прямая d
\newcommand{\ind}[1]{\mathbbm{1}_{\{#1\}}} % Индикатор события
%\renewcommand{\to}{\rightarrow}
\newcommand{\eqdef}{\mathrel{\stackrel{\rm def}=}}
\newcommand{\iid}{\mathrel{\stackrel{\rm i.\,i.\,d.}\sim}}
\newcommand{\const}{\mathrm{const}}
\usepackage[inline]{enumitem}

% вместо горизонтальной делаем косую черточку в нестрогих неравенствах
\renewcommand{\le}{\leqslant}
\renewcommand{\ge}{\geqslant}
\renewcommand{\leq}{\leqslant}
\renewcommand{\geq}{\geqslant}

\usepackage[backend=biber,bibencoding=utf8,sorting=nty,maxcitenames=4,style=numeric-verb]{biblatex}

\addbibresource{lit.bib}
\usepackage{csquotes}

\usepackage{multicol}
\usepackage{enumitem}

\renewcommand{\rmdefault}{cmss}
%\renewcommand{\ttdefault}{cmss}
\usepackage{sfmath}

\usepackage{pgfplotstable}
  
\begin{document}
	
	\maketitle
	
\section{Алгебра матриц}

\subsection{}
Определите размер следующих векторов и матриц:
\begin{multicols}{3}
	\begin{enumerate}[label=\alph*)]
		\item $H_{n\times k}$
		\item $\begin{pmatrix}
		1 \\ 2 \\ 3
		\end{pmatrix}$
		\item $\begin{pmatrix}
		4 & 5 & 6
		\end{pmatrix}$
		
		\item $\begin{pmatrix}
			a & b \\
			c & d \\
			k & m
		\end{pmatrix}$
		
		\item $\begin{pmatrix}
			1 & 1 & 1 \\
			2 & 2 & 2 \\
			3 & 3 & 3
		\end{pmatrix}$
		
		\item $I_{n\times n}$
	\end{enumerate}
\end{multicols}

\subsection{}
Вычислите:
\begin{multicols}{2}
		\begin{enumerate}[label=\alph*)]
		\item $\begin{pmatrix}
		1 \\ 2 \\ 4
		\end{pmatrix} + \begin{pmatrix}
		4 \\ 1 \\ 2
		\end{pmatrix}$
		
		\item $\begin{pmatrix}
		0 \\ 2 \\ 3
		\end{pmatrix} + 2\begin{pmatrix}
		3 \\ 1 \\ 8
		\end{pmatrix}$
		
		\item $ 0_{4 \times 1} + \begin{pmatrix}
		1 \\ 1  \\ 1 \\ 1
		\end{pmatrix}$
		
		\item $3\begin{pmatrix}
		7 \\ -1
		\end{pmatrix} - 2 \begin{pmatrix}
		1 \\ 4
		\end{pmatrix}$
	\end{enumerate}
\end{multicols}

\subsection{}
Вычислите:
\begin{multicols}{2}
	\begin{enumerate}[label=\alph*)]
		\item $\begin{pmatrix}
		1 & 1 \\
		2 & 7 
		\end{pmatrix} + \begin{pmatrix}
		3 & 1 \\
		0 & 3
		\end{pmatrix}$
		
		\item $\begin{pmatrix}
		4 & 4 & 1 \\
		2 & 0 & 9 \\
		1 & 1 & 0
		\end{pmatrix} - 3\begin{pmatrix}
		0 & 0 & 1 \\
		9 & 1 & 2 \\
		1 & 0 & 1
		\end{pmatrix}$
		
		\item $ I + \begin{pmatrix}
		1 & 4 \\
		2 & 2
		\end{pmatrix}$
		
		\item $a \begin{pmatrix}
		3 & 0 & 0 \\
		0 & 1 & 1 \\
		0 & 2 & 4 \\
		\end{pmatrix} + kI$.
	\end{enumerate}
\end{multicols}

\subsection{}
Вычислите матричное произведение, если это возможно:
\begin{multicols}{2}
	\begin{enumerate}[label=\alph*)]
		
		\item $I \times H$
		
		\item $\begin{pmatrix}
			1 & 3 \\
			2 & 1 
		\end{pmatrix} \times \begin{pmatrix}
		2 \\ 4
		\end{pmatrix}$
		
		\item $\begin{pmatrix}
		7 & 4 & 1 \\
		1 & 2 & 2
		\end{pmatrix} \times \begin{pmatrix}
		3 & 2 \\
		1 & 1 \\
		0 & 4
		\end{pmatrix}$
		
		\item $\begin{pmatrix}
		1 & 1 \\
		2 & 3
		\end{pmatrix} \times \begin{pmatrix}
		1 & 3
		\end{pmatrix}$
		
		\item $7\begin{pmatrix}
		3 & 3 \\
		1 & 2
		\end{pmatrix}\times I \times \begin{pmatrix}
		7 & 1 \\
		0 & 2
		\end{pmatrix}$
		
		\item $\begin{pmatrix}
		3 & 0 \\
		0 & 3
		\end{pmatrix}\times \begin{pmatrix}
		1 & 4 \\
		2 & 1
		\end{pmatrix}$
	\end{enumerate}
\end{multicols}

\section{Определитель}

\subsection{}
Найдите определитель следующих матриц, если это возможно:
	\begin{multicols}{4}
		\begin{enumerate}[label=\alph*)]
			\item $\begin{pmatrix}
			2 & 4 \\
			6 & 1
			\end{pmatrix}$
			
			\item $\begin{pmatrix}
			9 & 2 \\
			0 & 1
			\end{pmatrix}$
			
			\item $\begin{pmatrix}
			1 \\
			1
			\end{pmatrix}$
			
			\item $(1)$
			
			\item $I$
			
			\item $\begin{pmatrix}
			7 & 0 \\
			1 & 2
			\end{pmatrix}$
			
			\item $\begin{pmatrix}
			a & b \\
			c & d
			\end{pmatrix}$
			
			\item $\begin{pmatrix}
			1 & 1 \\
			2 & 2
			\end{pmatrix}$
		\end{enumerate}
	\end{multicols}

\subsection{}
Найдите матрицу, обратную данной, если это возможно:
\begin{multicols}{3}
	\begin{enumerate}[label=\alph*)]
		\item $\begin{pmatrix}
		1 & 2 \\
		3 & 4
		\end{pmatrix}$
		\item $\begin{pmatrix}
		1 & 1 \\
		2 & 2
		\end{pmatrix}$
		\item $\begin{pmatrix}
		4
		\end{pmatrix}$
		\item $\begin{pmatrix}
		1 & 0 & 0 \\
		0 & 3 & 0 \\
		0 & 0 & 4 
		\end{pmatrix}$
		\item $\begin{pmatrix}
			7 & 1 \\
			0 & 2
		\end{pmatrix}$
		\item $\begin{pmatrix}
			6 & 1 & 2 \\
			3 & 1 & 0
		\end{pmatrix}$
	\end{enumerate}
\end{multicols}

\subsection{}
Определите, являются ли векторы линейно зависимыми. Если дана матрица, определите, есть ли в ней линейно зависимые векторы.
\begin{multicols}{3}
	\begin{enumerate}[label=\alph*)]
		\item $\begin{pmatrix}
		1 \\ 0
		\end{pmatrix}$ и $\begin{pmatrix}
		0 \\ 1
		\end{pmatrix}$
		
		\item $\begin{pmatrix}
		1 \\ 1
		\end{pmatrix}$ и $\begin{pmatrix}
		3 \\ 3
		\end{pmatrix}$
		
		\item $\begin{pmatrix}
		3 \\ 2 \\ 3
		\end{pmatrix}$ и $\begin{pmatrix}
		27 \\ 14 \\ 27
		\end{pmatrix}$
		
		\item $\begin{pmatrix}
		3 \\ 2 \\ 3
		\end{pmatrix}$ и $\begin{pmatrix}
		9 \\ 6 \\ 9
		\end{pmatrix}$
		
		\item $\begin{pmatrix}
		3 & 1 & 2 \\
		3 & 4 & 5 \\
		18 & 6 & 12
		\end{pmatrix}$
		
		\item $\begin{pmatrix}
		1 & 2 \\
		2 & 1
		\end{pmatrix}$
		
		\item $\begin{pmatrix}
		1 & 1 & 0 \\
		0 & 3 & 0 \\
		1 & 12 & 0
		\end{pmatrix}$
		
		\item $\begin{pmatrix}
		1 & 1 & 3 \\
		4 & 2 & 6
		\end{pmatrix}$
	\end{enumerate}
\end{multicols}
	
\section{Векторы в пространстве}
	\subsection{}
	Вычислите скалярное произведение векторов:
	\begin{multicols}{3}
		\begin{enumerate}[label=\alph*)]
			
			\item $<\begin{pmatrix}
			1 \\ 1
			\end{pmatrix}, \begin{pmatrix}
			2 \\ 4
			\end{pmatrix}>$
			
			\item $<\begin{pmatrix}
			a \\ m \\ d
			\end{pmatrix}, \begin{pmatrix}
			c \\ k \\ l
			\end{pmatrix}>$
			
			\item $<a, b>$
		\end{enumerate}
	\end{multicols}

	\subsection{}
	Вычислите:
	\begin{multicols}{2}
		\begin{enumerate}[label=\alph*)]
			\item $a = \begin{pmatrix}
			1 \\
			2
			\end{pmatrix}, ||a|| = ?$
			\item $b = \begin{pmatrix}
			5 \\
			2 \\ 
			1
			\end{pmatrix}, ||b||^2 = ?$
			\item $c = \begin{pmatrix}
			c_1 \\
			c_2 \\
			\vdots \\
			c_n
			\end{pmatrix}, ||c|| = ?$
			\item $d = \begin{pmatrix}
			1 \\
			0 \\
			0 \\
			\end{pmatrix}, ||d||^2 = ?$
		\end{enumerate}
	\end{multicols}

	
	\subsection{}	
	Найдите косинус угла между векторами. Определите, являются ли векторы ортогональными. 
	\[
	\cos(\angle x, y) = \dfrac{<x, y>}{||x||\times||y||}
	\] 
	
	\begin{multicols}{3}
		\begin{enumerate}[label=\alph*)]
			\item $\begin{pmatrix}
			1 \\
			2 \\
			4
			\end{pmatrix}$, $\begin{pmatrix}
			2 \\
			1 \\
			0
			\end{pmatrix}$
			\item $\begin{pmatrix}
			1 \\
			2
			\end{pmatrix}$, $\begin{pmatrix}
			-1 \\
			0
			\end{pmatrix}$
			\item $\begin{pmatrix}
			3 \\
			4 \\
			1
			\end{pmatrix}$, $\begin{pmatrix}
			1 \\
			-1 \\
			1
			\end{pmatrix}$
			\item $\begin{pmatrix}
			1 \\
			2 \\
			1
			\end{pmatrix}$, $\begin{pmatrix}
			-1 \\
			1 \\
			0
			\end{pmatrix}$
			\item $\begin{pmatrix}
			-1 \\
			-2 \\
			6
			\end{pmatrix}$, $\begin{pmatrix}
			4 \\
			-2 \\
			0
			\end{pmatrix}$
			\item $\begin{pmatrix}
			0 \\
			-2 \\
			4 \\
			1
			\end{pmatrix}$, $\begin{pmatrix}
			12 \\
			-1 \\
			-1 \\
			2
			\end{pmatrix}$
		\end{enumerate}
	\end{multicols}

	\subsection{}
	Изобразите следующие векторы и системы векторов. Определите, содержит ли система векторов линейно зависимые векторы. 
		\begin{multicols}{2}
		\begin{enumerate}[label=\alph*)]
			\item $\begin{pmatrix}
			1 \\
			2
			\end{pmatrix}$
			\item $\begin{pmatrix}
			1 \\
			0
			\end{pmatrix}$, 
			$\begin{pmatrix}
			0 \\
			1
			\end{pmatrix}$
			\item$\begin{pmatrix}
			5 & 1 & 0 \\
			4 & 0 & 1
			\end{pmatrix}$
			\item $\begin{pmatrix}
			4 & 2 \\
			1 & 0.5 \\
			2 & 0
			\end{pmatrix}$
			\item $\begin{pmatrix}
			1 & 2 & 0 \\
			1 & 2 & 1 \\
			1 & 2 & 0 \\
			\end{pmatrix}$
			\item $y_n$, $x_n$, $<y, x> = 0$
			\item $y_k$, $x_k$, $z_k$, $||x||^2 + ||y||^2 = ||z||^2$
			\item $y_{100}$, $p_{100}$, $m_{100}$, $m_{100} = y_{100} + 2p_{100}$
		\end{enumerate}
	\end{multicols}
	
	\subsection{}
	Изобразите проекцию вектора $Y$ на указанное пространство. В пункте a) рассчитайте координаты проекции.
		\begin{enumerate}[label=\alph*)]
		\item $Y = \begin{pmatrix}
		3 \\
		1 \\
		2 \\
		\end{pmatrix}$ на $a = \begin{pmatrix}
		1 \\
		1 \\
		1 \\
		\end{pmatrix}$
		
		\item $Y = \begin{pmatrix}
		y_1 \\
		\vdots \\
		y_n
		\end{pmatrix}$ на $X = \begin{pmatrix}
		x_{11} & x_{12} \\
		\vdots & \vdots \\
		x_{n1} & x_{n2}
		\end{pmatrix}$
		
		\item $Y = \begin{pmatrix}
		y_1 \\
		\vdots \\
		y_n
		\end{pmatrix}$ на $X_{n \times k}$
		\end{enumerate}
	
\end{document}