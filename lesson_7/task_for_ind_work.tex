\documentclass[11pt, a4paper]{extarticle}
\setlength{\parskip}{0.2em}
%%% Работа с русским языком
\usepackage{cmap}					% поиск в PDF
\usepackage{mathtext} 				% русские буквы в формулах
\usepackage[T2A]{fontenc}			% кодировка
\usepackage[utf8]{inputenc}			% кодировка исходного текста
\usepackage[english,russian]{babel}	% локализация и переносы
\usepackage{mathtools}   % loads »amsmath«
\usepackage{graphicx}
\usepackage{caption}
\usepackage{physics}
\usepackage{subcaption}
\usepackage{tikz}
\usepackage{multicol}
\usepackage{listings}
\usepackage{tabularx}

\usepackage{hyperref}
\hypersetup{				% Гиперссылки
    unicode=true,           % русские буквы в раздела PDF
    pdftitle={Заголовок},   % Заголовок
    pdfauthor={Автор},      % Автор
    pdfsubject={Тема},      % Тема
    pdfcreator={Создатель}, % Создатель
    pdfproducer={Производитель}, % Производитель
    pdfkeywords={keyword1} {key2} {key3}, % Ключевые слова
    colorlinks=true,       	% false: ссылки в рамках; true: цветные ссылки
    linkcolor=black,          % внутренние ссылки
    citecolor=black,        % на библиографию
    filecolor=cyan,      % на файлы
    urlcolor=magenta           % на URL
}

\usepackage{setspace} % Интерлиньяж
%\onehalfspacing % Интерлиньяж 1.5
%\doublespacing % Интерлиньяж 2
\singlespacing % Интерлиньяж 1

%%% Дополнительная работа с математикой
\usepackage{amsmath,amsfonts,amssymb,amsthm,mathtools} % AMS
\usepackage{icomma} % "Умная" запятая: $0,2$ --- число, $0, 2$ --- перечисление

%% Шрифты
\usepackage{euscript}	 % Шрифт Евклид
\usepackage{mathrsfs} % Красивый матшрифт
\usepackage{float}

\title{Практическое занятие 1}

\author{Факультатив «Введение в анализ данных и машинное обучение на Python»}
\date{30 ноября 2019 г.}

\usepackage{geometry}
\geometry{
	a4paper,
	left=20mm,
	top=20mm,
	right=20mm
}
%\setlength{\parindent}{0cm}

\DeclareMathOperator{\Lin}{\mathrm{Lin}}
\DeclareMathOperator{\Linp}{\Lin^{\perp}}
\DeclareMathOperator*\plim{plim}
%\DeclareMathOperator{\grad}{grad}
\DeclareMathOperator{\card}{card}
\DeclareMathOperator{\sgn}{sign}
\DeclareMathOperator{\sign}{sign}

\DeclareMathOperator*{\argmin}{arg\,min}
\DeclareMathOperator*{\argmax}{arg\,max}
\DeclareMathOperator*{\amn}{arg\,min}
\DeclareMathOperator*{\amx}{arg\,max}
\DeclareMathOperator{\cov}{Cov}
\DeclareMathOperator{\Var}{Var}
\DeclareMathOperator{\Cov}{Cov}
\DeclareMathOperator{\Corr}{Corr}
\DeclareMathOperator{\pCorr}{pCorr}
\DeclareMathOperator{\E}{\mathbb{E}}
\let\P\relax
\DeclareMathOperator{\P}{\mathbb{P}}



\newcommand{\cN}{\mathcal{N}}
\newcommand{\cU}{\mathcal{U}}
\newcommand{\cBinom}{\mathcal{Binom}}
\newcommand{\cPois}{\mathcal{Pois}}
\newcommand{\cBeta}{\mathcal{Beta}}
\newcommand{\cGamma}{\mathcal{Gamma}}

\def \R{\mathbb{R}}
\def \N{\mathbb{N}}
\def \Z{\mathbb{Z}}

\usepackage{multirow}
\usepackage{array}


\newcommand{\dx}[1]{\,\mathrm{d}#1} % для интеграла: маленький отступ и прямая d
\newcommand{\ind}[1]{\mathbbm{1}_{\{#1\}}} % Индикатор события
%\renewcommand{\to}{\rightarrow}
\newcommand{\eqdef}{\mathrel{\stackrel{\rm def}=}}
\newcommand{\iid}{\mathrel{\stackrel{\rm i.\,i.\,d.}\sim}}
\newcommand{\const}{\mathrm{const}}
\usepackage[inline]{enumitem}

% вместо горизонтальной делаем косую черточку в нестрогих неравенствах
\renewcommand{\le}{\leqslant}
\renewcommand{\ge}{\geqslant}
\renewcommand{\leq}{\leqslant}
\renewcommand{\geq}{\geqslant}

\usepackage[backend=biber,bibencoding=utf8,sorting=nty,maxcitenames=4,style=numeric-verb]{biblatex}

\addbibresource{lit.bib}
\usepackage{csquotes}

\usepackage{multicol}
\usepackage{enumitem}

\renewcommand{\rmdefault}{cmss}
%\renewcommand{\ttdefault}{cmss}
\usepackage{sfmath}

\usepackage{pgfplotstable}
  
\begin{document}
	
	\maketitle
	
\section{Общие комментарии}
\begin{enumerate}
	\item На этом занятии Вам предстоит самостоятельно пройти первые шаги работы специалиста по машинному обучению: предварительный анализ набора данных, постановка возможных задач и решение одной из них. Задания, описанные здесь, помогут Вам сделать это.
	\item Работа аналитика данных очень творческая, поэтому на приведённые ниже вопросы нет единственного правильного ответа. Если ответ кажется Вам интуитивно правильным, напишите его. Если Вам кажется, что на вопрос можно дать несколько ответов, опишите их все.
	\item Для выполнения задания можно использовать: 
	\begin{itemize}
		\item Все тетрадки, презентации и другие материалы предыдущих занятий.
		\item Любые ресурсы в Интернете, документацию, форумы и проч. 
		\item Любые печатные материалы и проч. 
	\end{itemize}
	Однако, не прибегайте, пожалуйста, к помощи Ваших коллег (более формально: задание выполняется индивидуально). Почему – см. пункты 7 и 8.
	\item Если вы копируете код с минимальными изменениями, не забывайте указывать источник (в виде комментария Python в ячейке с кодом).
	\item Для работы Вам необходимо создать пустую тетрадку в Jupyter Notebook. Все задания необходимо выполнять в этой тетрадке. Все смысловые комментарии и текстовые пояснения следует писать в ячейках {\tt Markdown}. Если Вы не помните, как работать с ячейками {\tt Markdown}, откройте тетрадку с первого занятия и посмотрите, как эти ячейки устроены там (напоминание: чтобы посмотреть содержимое ячейки, щёлкните по ней левой кнопкой мыши два раза). 
	\item Оформление важно. В первой ячейке тетрадки сделайте заголовок Вашей работы по типу того, как это устроено в тетрадках с предыдущих занятий. Во второй ячейке напишите свои фамилию и имя. При выполнении заданий указывайте номер задания (По типу «2.1»). У Вашего файла должна быть понятная структура. Не забудьте про оформление графиков.
	\item Это задание \textit{не на оценку}. Выполняя его, Вы сможете лучше понять пройденный материал, а также потренироваться для выполнения домашнего задания. Постарайтесь сделать как можно больше заданий, но не страшно, если Вы не успеете сделать все. 
	\item Тетрадку с выполненным заданием необходимо прислать на почту vsomelyusik@gmail.com
	
	Это нужно сделать для того, чтобы получить комментарии, которые помогут Вам в выполнении текущего и будущего домашних заданий. 
\end{enumerate}

\section{Разминка}
Так как мы не закончили выполнение дополнительных заданий из первого занятия, сейчас самое время сделать их! 
\begin{enumerate}
	\item Откройте тетрадку из «Темы 1: Введение в Python». Скопируйте задания и заготовки кода из «Части 7: Дополнительные задания» в Вашу тетрадку.
	\item Выполните эти задания. 
\end{enumerate}

\section{Анализ данных}
\begin{enumerate}
	\item Импортируйте необходимые библиотеки.
	\item Импортируйте данные к занятию и представьте их в виде {\tt dataframe}. Изучите описание данных по \href{https://www.kaggle.com/budincsevity/szeged-weather}{ссылке}. Кратко опишите, как Вы поняли, что из себя представляют данные. 
	\item Предложите две задачи регрессии и две задачи классификации, которые можно бы было поставить для этого набора данных (задачи сформулируйте в виде вопросов, например, «Как цена на жильё зависит от среднего числа комнат в квартире и уровня загрязнения воздуха в районе?» – задача регрессии). 
	\item На этом занятии Вам необходимо решить задачу регрессии, в которой зависимую переменную необходимо объяснить не более чем тремя, но не менее чем двумя, независимыми переменными. Выберите одну из формулировок задачи из предыдущего пункта или предложите новую. 
	\item С учётом предыдущего пункта, выберите в данных переменную, которую Вы будете считать зависимой (то есть которую Вы будете предсказывать), а также выберите независимые переменные (две или три). Помните о том, какого типа должна быть зависимая переменная в задаче регрессии. 
	\item Создайте новый {\tt dataframe}, в который будут входить только выбранные в предыдущем пункте переменные (и зависимая, и независимые). 
	
	\vspace{1.5em}
	В дальнейших заданиях речь идёт о данных, которые вы сформировали сами в пункте 6.
	
	\item Каково число наблюдений в данных? 
	\item Если считаете нужным, измените названия столбцов Вашего набора данных. Определите типы каждой независимой переменной.  
	\item Есть ли в данных пропущенные значения? Если да, то сколько их? Обработайте пропущенные значения так, как считаете нужным. Сколько наблюдений в данных теперь?
	\item Постройте гистограммы всех переменных. Проинтерпретируйте результаты.
	\item Постройте корреляционную матрицу переменных. Что показывает корреляция? Какие независимые переменные наиболее связаны с зависимой? 
	\item Постройте диаграммы рассеяния зависимой переменной (по оси Y) против каждой независимой переменной (по оси X). Проинтерпретируйте результат. Согласуются ли выводы с выводами предыдущего пункта? 
	\item Какой вид зависимости наблюдается для каждой независимой переменной (линейная / нелинейная, возрастающая / убывающая)? Приведите возможное объяснение наблюдаемым зависимостям.
\end{enumerate}

\section{Построение простой модели}
\begin{enumerate}
	\item Мы будем строить модель вида:
	\[
	\hat{Y}_i = \hat{\beta}_0 + \hat{\beta}_1X_{1i},
	\]
	где $Y_i$ – зависимая переменная, $X_1$ – объясняющая переменная, имеющая наибольшую корреляцию с зависимой переменной, $\hat{\beta}_0$, $\hat{\beta}_1$ – некоторые коэффициенты.
	
	Как называется данный вид регрессии?
	
	\item С математической точки зрения, что означает выписанное уравнение модели (то есть какой математический объект представляет собой это уравнение)?
	\item Подробности того, как с теоретической точки зрения построить данную линию, мы узнаем на следующем занятии. Пока же мы доверимся библиотеке {\tt sklearn}, которую также подробно разберём на следующих занятиях. Выполните следующую команду:
	
	 {\tt from sklearn.linear\_model import LinearRegression}
	 
	  – и поясните, что она делает. 
	
	\item Выполните следующую команду: 
	
	{\tt model = LinearRegression()}
	
	 где {\tt model} – это название переменной, а {\tt LinearRegression()} – класс, в котором содержится реализация используемой модели. Вспомните второе занятие и напишите, чем с точки зрения Python является переменная {\tt model}.
	
	\item Обучим модель! Обучение в данном случае заключается в подборе коэффициентов  $\hat{\beta}_0$ и $\hat{\beta}_1$. Для обучения модели используется команда:
	
	 {\tt model.fit(X, Y)}
	 
	  где вместо {\tt X} передаётся матрица объясняющих переменных, вместо {\tt Y} – столбец зависимых переменных. Обычно переменные можно передавать обычным индексированием столбцов из {\tt dataframe}, однако в случае одной объясняющей переменной могут возникнуть проблемы с размерностью. Тогда объясняющую переменную нужно представить в виде массива {\tt numpy} правильной размерности. Пока не вдаваясь в подробности, для переменной {\tt Humidity} это можно сделать так: 
	  
	  {\tt np.array(data['Humidity']).reshape(-1, 1)}
	  
	   Тогда полный код обучения модели будет выглядеть следующим образом: 
	   
	    {\tt model.fit(np.array(data['Humidity']).reshape(-1, 1), data['Temperature (C)'])}
	    
	     где {\tt Temperature (C)} – зависимая переменная.
	
	Повторите эту процедуру для Вашей модели. 
	
	\item Для получения обученных коэффициентов воспользуйтесь командами: 
	{\tt model.coef\_} – для коэффициента наклона $\hat{\beta}_1$, {\tt model.intercept\_} – для коэффициента сдвига $\hat{\beta}_0$. Проинтерпретируйте знаки коэффициентов. Соотносятся ли они с выводами об этой объясняющей переменной, полученными из предыдущего анализа? 
	\item Постройте на одном графике:
	\begin{enumerate}[label=\alph*)]
		\item Диаграмму рассеяния $\hat{Y}_i$ и $X_{1i}$. 
		\item Уравнение обученной модели (задайте его вручную, используя полученные коэффициенты).
	\end{enumerate}
	Как Вы считаете, насколько хорошо модель соответствует данным? Велика ли её обобщающая способность? 
	
\end{enumerate}

\section{Построение более сложной модели}

\begin{enumerate}
	\item В этой части мы будем строить модель вида:
	\[
	\hat{Y}_i = \hat{\beta}_0 + \hat{\beta}_1X_{1i} +  \hat{\beta}_2X_{2i} + \hat{\beta}_3X_{3i},
	\]
	где $Y_i$ – зависимая переменная, $X_i$ – объясняющие переменные (понятно, что это общий вид модели, и их количество равно тому, сколько Вы выбрали объясняющих переменной для Вашей модели), $\hat{\beta}$ – некоторые коэффициенты. 
	
	\item Выполните команду: {\tt model2 = LinearRegression()}. По аналогии с пунктом 4 из предыдущего задания, поясните, что делает данная команда. 
	
	\item Адаптируйте следующую команду для Вашей модели: 
	
	 {\tt model2.fit(data.loc[:, ['Humidity', 'Visibility (km)']], data['Temperature (C)'])}
	 
	  По аналогии с пунктом 5 из предыдущего задания, поясните, что делает данная команда. 
	
	\item По аналогии с пунктом 6 из предыдущего задания, получите коэффициенты обученной модели (коэффициенты в списке идут в соответствии с порядком независимых переменных, заданном при обучении модели). 
	
	\item Повторите пункт 7 из предыдущего задания для каждой независимой переменной в модели из этого задания. 
\end{enumerate}


\end{document}