\documentclass[11pt, a4paper]{extarticle}
\setlength{\parskip}{0.2em}
%%% Работа с русским языком
\usepackage{cmap}					% поиск в PDF
\usepackage{mathtext} 				% русские буквы в формулах
\usepackage[T2A]{fontenc}			% кодировка
\usepackage[utf8]{inputenc}			% кодировка исходного текста
\usepackage[english,russian]{babel}	% локализация и переносы
\usepackage{mathtools}   % loads »amsmath«
\usepackage{graphicx}
\usepackage{caption}
\usepackage{physics}
\usepackage{subcaption}
\usepackage{tikz}
\usepackage{multicol}
\usepackage{listings}
\usepackage{tabularx}

\usepackage{hyperref}
\hypersetup{				% Гиперссылки
    unicode=true,           % русские буквы в раздела PDF
    pdftitle={Заголовок},   % Заголовок
    pdfauthor={Автор},      % Автор
    pdfsubject={Тема},      % Тема
    pdfcreator={Создатель}, % Создатель
    pdfproducer={Производитель}, % Производитель
    pdfkeywords={keyword1} {key2} {key3}, % Ключевые слова
    colorlinks=true,       	% false: ссылки в рамках; true: цветные ссылки
    linkcolor=black,          % внутренние ссылки
    citecolor=black,        % на библиографию
    filecolor=cyan,      % на файлы
    urlcolor=gray           % на URL
}

\usepackage{setspace} % Интерлиньяж
%\onehalfspacing % Интерлиньяж 1.5
%\doublespacing % Интерлиньяж 2
\singlespacing % Интерлиньяж 1

%%% Дополнительная работа с математикой
\usepackage{amsmath,amsfonts,amssymb,amsthm,mathtools} % AMS
\usepackage{icomma} % "Умная" запятая: $0,2$ --- число, $0, 2$ --- перечисление

%% Шрифты
\usepackage{euscript}	 % Шрифт Евклид
\usepackage{mathrsfs} % Красивый матшрифт
\usepackage{float}

\title{Линейные методы регрессии II}

\author{Факультатив «Введение в анализ данных и машинное обучение на Python»}
\date{\today}

\usepackage{geometry}
\geometry{
	a4paper,
	left=20mm,
	top=20mm,
	right=20mm
}
\setlength{\parindent}{0cm}

\DeclareMathOperator{\Lin}{\mathrm{Lin}}
\DeclareMathOperator{\Linp}{\Lin^{\perp}}
\DeclareMathOperator*\plim{plim}
%\DeclareMathOperator{\grad}{grad}
\DeclareMathOperator{\card}{card}
\DeclareMathOperator{\sgn}{sign}
\DeclareMathOperator{\sign}{sign}

\DeclareMathOperator*{\argmin}{arg\,min}
\DeclareMathOperator*{\argmax}{arg\,max}
\DeclareMathOperator*{\amn}{arg\,min}
\DeclareMathOperator*{\amx}{arg\,max}
\DeclareMathOperator{\cov}{Cov}
\DeclareMathOperator{\Var}{Var}
\DeclareMathOperator{\Cov}{Cov}
\DeclareMathOperator{\Corr}{Corr}
\DeclareMathOperator{\pCorr}{pCorr}
\DeclareMathOperator{\E}{\mathbb{E}}
\let\P\relax
\DeclareMathOperator{\P}{\mathbb{P}}



\newcommand{\cN}{\mathcal{N}}
\newcommand{\cU}{\mathcal{U}}
\newcommand{\cBinom}{\mathcal{Binom}}
\newcommand{\cPois}{\mathcal{Pois}}
\newcommand{\cBeta}{\mathcal{Beta}}
\newcommand{\cGamma}{\mathcal{Gamma}}

\def \R{\mathbb{R}}
\def \N{\mathbb{N}}
\def \Z{\mathbb{Z}}

\usepackage{multirow}
\usepackage{array}


\newcommand{\dx}[1]{\,\mathrm{d}#1} % для интеграла: маленький отступ и прямая d
\newcommand{\ind}[1]{\mathbbm{1}_{\{#1\}}} % Индикатор события
%\renewcommand{\to}{\rightarrow}
\newcommand{\eqdef}{\mathrel{\stackrel{\rm def}=}}
\newcommand{\iid}{\mathrel{\stackrel{\rm i.\,i.\,d.}\sim}}
\newcommand{\const}{\mathrm{const}}
\usepackage[inline]{enumitem}

% вместо горизонтальной делаем косую черточку в нестрогих неравенствах
\renewcommand{\le}{\leqslant}
\renewcommand{\ge}{\geqslant}
\renewcommand{\leq}{\leqslant}
\renewcommand{\geq}{\geqslant}

\usepackage[backend=biber,bibencoding=utf8,sorting=nty,maxcitenames=4,style=numeric-verb]{biblatex}

\addbibresource{lit.bib}
\usepackage{csquotes}

\usepackage{multicol}
\usepackage{enumitem}

\renewcommand{\rmdefault}{cmss}
%\renewcommand{\ttdefault}{cmss}
\usepackage{sfmath}

\usepackage{pgfplotstable}
  
\begin{document}
	
	\maketitle
	
\section*{Задание}
Пусть представлены следующие данные:
\[
Y = \begin{pmatrix}
2 \\
4 \\
8 \\
\end{pmatrix}, \text{ } X = \begin{pmatrix}
1 & 1 \\
1 & 2 \\
1 & 2
\end{pmatrix}
\]
\begin{enumerate}[label=\alph*)]
	\item Оценивается следующая модель:
	\[Y = X\beta + u,\]
	где $u$ – случайная ошибка.
	
	Найдите оценки коэффициентов модели. Найдите прогноз модели.
	\item Нанесите выборку на диаграмму рассеяния. Постройте оценённую линию регрессии. 
	\item Определите направление корреляции между $X$ и $Y$.
	\item Найдите выборочное среднее и выборочную дисперсию неконстантного признака.  
	\item Найдите коэффициент детерминации данной регрессии. 
\end{enumerate}

\section*{Решение}
\begin{enumerate}[label=\alph*)]
	\item
	\begin{align*}
		\hat{\beta} &= (X^TX)^{-1}X^TY \\
		X^TX &= \begin{pmatrix}
		3 & 5 \\
		5 & 9
		\end{pmatrix} \\
		(X^TX)^{-1} &= \dfrac{1}{2}\begin{pmatrix}
		9 & \text{-}5 \\
		\text{-}5 & 3
		\end{pmatrix} \\
		\hat{\beta} = \dfrac{1}{2}\begin{pmatrix}
		9 & \text{-}5 \\
		\text{-}5 & 3
		\end{pmatrix}&\begin{pmatrix}
		1 & 1 & 1 \\
		1 & 2 & 2
		\end{pmatrix}\begin{pmatrix}
		2 \\ 
		4 \\
		8
		\end{pmatrix} = \begin{pmatrix}
		\text{-}2 \\
		4
		\end{pmatrix} \\
		\hat{Y} &= X\hat{\beta} = \begin{pmatrix}
		2 \\
		6 \\
		6
		\end{pmatrix}
	\end{align*}
	\item Строим в координатах $(X_1, Y)$, где $X_1$ – второй столбец матрицы $X$. Уравнение оценённой линии: $\hat{Y} = -2 + 4X_1$. Можно построить и без уравнения, последовательно нанося точки $(X_1, \hat{Y})$. 
	\item Положительное. 
	\item 
	\begin{align*}
		\bar{X} =& \dfrac{1 + 2 + 2}{3} = \dfrac{5}{3} \\\\
		s\Var(X) =& (1 - \frac{5}{3})^2 + (2 - \frac{5}{3})^2 + (2 - \frac{5}{3})^2.
	\end{align*}
	
	\item $R^2 = 1 - \dfrac{||Y - \hat{Y}||^2}{||Y - \bar{Y}||^2} = 1 - \dfrac{0^2 + (-2)^2 + 2^2}{(2 - 14/3)^2 + (4 - 14/3)^2 + (8 - 14/3)^2} = 0.57.$
\end{enumerate}

	
\end{document}