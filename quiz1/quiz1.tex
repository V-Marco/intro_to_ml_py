\documentclass[11pt, a4paper]{extarticle}
\setlength{\parskip}{0.2em}
%%% Работа с русским языком
\usepackage{cmap}					% поиск в PDF
\usepackage{mathtext} 				% русские буквы в формулах
\usepackage[T2A]{fontenc}			% кодировка
\usepackage[utf8]{inputenc}			% кодировка исходного текста
\usepackage[english,russian]{babel}	% локализация и переносы
\usepackage{mathtools}   % loads »amsmath«
\usepackage{graphicx}
\usepackage{caption}
\usepackage{physics}
\usepackage{subcaption}
\usepackage{tikz}
\usepackage{multicol}
\usepackage{listings}
\usepackage{tabularx}

\usepackage{hyperref}
\hypersetup{				% Гиперссылки
    unicode=true,           % русские буквы в раздела PDF
    pdftitle={Заголовок},   % Заголовок
    pdfauthor={Автор},      % Автор
    pdfsubject={Тема},      % Тема
    pdfcreator={Создатель}, % Создатель
    pdfproducer={Производитель}, % Производитель
    pdfkeywords={keyword1} {key2} {key3}, % Ключевые слова
    colorlinks=true,       	% false: ссылки в рамках; true: цветные ссылки
    linkcolor=black,          % внутренние ссылки
    citecolor=black,        % на библиографию
    filecolor=cyan,      % на файлы
    urlcolor=gray           % на URL
}

\usepackage{setspace} % Интерлиньяж
%\onehalfspacing % Интерлиньяж 1.5
%\doublespacing % Интерлиньяж 2
\singlespacing % Интерлиньяж 1

%%% Дополнительная работа с математикой
\usepackage{amsmath,amsfonts,amssymb,amsthm,mathtools} % AMS
\usepackage{icomma} % "Умная" запятая: $0,2$ --- число, $0, 2$ --- перечисление

%% Шрифты
\usepackage{euscript}	 % Шрифт Евклид
\usepackage{mathrsfs} % Красивый матшрифт
\usepackage{float}

\title{Проверочная работа 1}

\author{Факультатив «Введение в анализ данных и машинное обучение на Python»}
\date{16 ноября 2019}

\usepackage{geometry}
\geometry{
	a4paper,
	left=20mm,
	top=20mm,
	right=20mm
}
%\setlength{\parindent}{0cm}

\DeclareMathOperator{\Lin}{\mathrm{Lin}}
\DeclareMathOperator{\Linp}{\Lin^{\perp}}
\DeclareMathOperator*\plim{plim}
%\DeclareMathOperator{\grad}{grad}
\DeclareMathOperator{\card}{card}
\DeclareMathOperator{\sgn}{sign}
\DeclareMathOperator{\sign}{sign}

\DeclareMathOperator*{\argmin}{arg\,min}
\DeclareMathOperator*{\argmax}{arg\,max}
\DeclareMathOperator*{\amn}{arg\,min}
\DeclareMathOperator*{\amx}{arg\,max}
\DeclareMathOperator{\cov}{Cov}
\DeclareMathOperator{\Var}{Var}
\DeclareMathOperator{\Cov}{Cov}
\DeclareMathOperator{\Corr}{Corr}
\DeclareMathOperator{\pCorr}{pCorr}
\DeclareMathOperator{\E}{\mathbb{E}}
\let\P\relax
\DeclareMathOperator{\P}{\mathbb{P}}



\newcommand{\cN}{\mathcal{N}}
\newcommand{\cU}{\mathcal{U}}
\newcommand{\cBinom}{\mathcal{Binom}}
\newcommand{\cPois}{\mathcal{Pois}}
\newcommand{\cBeta}{\mathcal{Beta}}
\newcommand{\cGamma}{\mathcal{Gamma}}

\def \R{\mathbb{R}}
\def \N{\mathbb{N}}
\def \Z{\mathbb{Z}}

\usepackage{multirow}
\usepackage{array}


\newcommand{\dx}[1]{\,\mathrm{d}#1} % для интеграла: маленький отступ и прямая d
\newcommand{\ind}[1]{\mathbbm{1}_{\{#1\}}} % Индикатор события
%\renewcommand{\to}{\rightarrow}
\newcommand{\eqdef}{\mathrel{\stackrel{\rm def}=}}
\newcommand{\iid}{\mathrel{\stackrel{\rm i.\,i.\,d.}\sim}}
\newcommand{\const}{\mathrm{const}}
\usepackage[inline]{enumitem}

% вместо горизонтальной делаем косую черточку в нестрогих неравенствах
\renewcommand{\le}{\leqslant}
\renewcommand{\ge}{\geqslant}
\renewcommand{\leq}{\leqslant}
\renewcommand{\geq}{\geqslant}

\usepackage[backend=biber,bibencoding=utf8,sorting=nty,maxcitenames=4,style=numeric-verb]{biblatex}

\addbibresource{lit.bib}
\usepackage{csquotes}

\usepackage{multicol}
\usepackage{enumitem}

\renewcommand{\rmdefault}{cmss}
%\renewcommand{\ttdefault}{cmss}
\usepackage{sfmath}

\usepackage{pgfplotstable}
  
\begin{document}
	
	\maketitle
	
	\section*{Задание 1} 
	\begin{enumerate}[label=\alph*)]
	\item Чем коллекции отличаются от типов данных в Python? 
	\item Является ли неизменяемое множество примером коллекции?
	\end{enumerate}

	\section*{Задание 2}
	\begin{enumerate}[label=\alph*)]
		\item Объясните своими словами, что такое генеральная совокупность и выборка.
		\item Что такое репрезентативность выборки?
		\item Какими свойствами должна обладать выборка, чтобы быть репрезентативной?
	\end{enumerate}
	
	\section*{Задание 3}
	Вычислите $A^{-1}$, если это возможно. Если обратную матрицу найти невозможно, объясните почему.
	\begin{multicols}{3}
	\begin{enumerate}[label=\alph*)]
		\item $A = \begin{pmatrix}
		3 & 3 & 3 \\
		100 & -100 & 3 \\
		27 & 27 & 27
		\end{pmatrix}$
		
		\item $A = \begin{pmatrix}
		1 & 2 & -3 & 4 \\
		-9 & 1 & 2 & -10
		\end{pmatrix}$
		
		\item $A = \begin{pmatrix}
		3 & -1 \\
		-2 & -4
		\end{pmatrix}$
	\end{enumerate} 
	\end{multicols}
	
	\section*{Задание 4}
	Найдите число наблюдений и число признаков и определите типы признаков в следующей таблице «объекты-признаки»:
	\begin{center}
		\def\arraystretch{1.5}
		\begin{tabular}{l| c c c c c c}
			\hline
			& $X_1$ & $X_2$ & $X_3$  & $X_4$ & $X_5$ & $X_6$ \\
			\hline
			$1$ & Мария & 24 & Москва & 1 & 0 & 0.25  \\
			$2$ & Дарья & 17 & Москва & 3 & 0 & 0.1  \\
			$3$ & Александр & 16 & Санкт-Петербург & 3 & 1 & 0.2  \\
			$4$ & Степан & 25 & Казань & 3 & 1 & 0.08  \\
			$5$ & Евгения & 29 & Москва & 2 & 0 & 0.01  \\
		\end{tabular}
	\end{center}
	{\small Примечание: переменная $X_4$ может принимать значения $1 < 2 < 3$, а переменная $X_5$ – значения 0 и 1.}
	
\end{document}