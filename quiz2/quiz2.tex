\documentclass[11pt, a4paper]{extarticle}
\setlength{\parskip}{0.2em}
%%% Работа с русским языком
\usepackage{cmap}					% поиск в PDF
\usepackage{mathtext} 				% русские буквы в формулах
\usepackage[T2A]{fontenc}			% кодировка
\usepackage[utf8]{inputenc}			% кодировка исходного текста
\usepackage[english,russian]{babel}	% локализация и переносы
\usepackage{mathtools}   % loads »amsmath«
\usepackage{graphicx}
\usepackage{caption}
\usepackage{physics}
\usepackage{subcaption}
\usepackage{tikz}
\usepackage{multicol}
\usepackage{listings}
\usepackage{tabularx}

\usepackage{hyperref}
\hypersetup{				% Гиперссылки
    unicode=true,           % русские буквы в раздела PDF
    pdftitle={Заголовок},   % Заголовок
    pdfauthor={Автор},      % Автор
    pdfsubject={Тема},      % Тема
    pdfcreator={Создатель}, % Создатель
    pdfproducer={Производитель}, % Производитель
    pdfkeywords={keyword1} {key2} {key3}, % Ключевые слова
    colorlinks=true,       	% false: ссылки в рамках; true: цветные ссылки
    linkcolor=black,          % внутренние ссылки
    citecolor=black,        % на библиографию
    filecolor=cyan,      % на файлы
    urlcolor=gray           % на URL
}

\usepackage{setspace} % Интерлиньяж
%\onehalfspacing % Интерлиньяж 1.5
%\doublespacing % Интерлиньяж 2
\singlespacing % Интерлиньяж 1

%%% Дополнительная работа с математикой
\usepackage{amsmath,amsfonts,amssymb,amsthm,mathtools} % AMS
\usepackage{icomma} % "Умная" запятая: $0,2$ --- число, $0, 2$ --- перечисление

%% Шрифты
\usepackage{euscript}	 % Шрифт Евклид
\usepackage{mathrsfs} % Красивый матшрифт
\usepackage{float}

\title{Проверочная работа 2}

\author{Факультатив «Введение в анализ данных и машинное обучение на Python»}
\date{21 декабря 2019}

\usepackage{geometry}
\geometry{
	a4paper,
	left=20mm,
	top=20mm,
	right=20mm
}
%\setlength{\parindent}{0cm}

\DeclareMathOperator{\Lin}{\mathrm{Lin}}
\DeclareMathOperator{\Linp}{\Lin^{\perp}}
\DeclareMathOperator*\plim{plim}
%\DeclareMathOperator{\grad}{grad}
\DeclareMathOperator{\card}{card}
\DeclareMathOperator{\sgn}{sign}
\DeclareMathOperator{\sign}{sign}

\DeclareMathOperator*{\argmin}{arg\,min}
\DeclareMathOperator*{\argmax}{arg\,max}
\DeclareMathOperator*{\amn}{arg\,min}
\DeclareMathOperator*{\amx}{arg\,max}
\DeclareMathOperator{\cov}{Cov}
\DeclareMathOperator{\Var}{Var}
\DeclareMathOperator{\Cov}{Cov}
\DeclareMathOperator{\Corr}{Corr}
\DeclareMathOperator{\pCorr}{pCorr}
\DeclareMathOperator{\E}{\mathbb{E}}
\let\P\relax
\DeclareMathOperator{\P}{\mathbb{P}}



\newcommand{\cN}{\mathcal{N}}
\newcommand{\cU}{\mathcal{U}}
\newcommand{\cBinom}{\mathcal{Binom}}
\newcommand{\cPois}{\mathcal{Pois}}
\newcommand{\cBeta}{\mathcal{Beta}}
\newcommand{\cGamma}{\mathcal{Gamma}}

\def \R{\mathbb{R}}
\def \N{\mathbb{N}}
\def \Z{\mathbb{Z}}

\usepackage{multirow}
\usepackage{array}


\newcommand{\dx}[1]{\,\mathrm{d}#1} % для интеграла: маленький отступ и прямая d
\newcommand{\ind}[1]{\mathbbm{1}_{\{#1\}}} % Индикатор события
%\renewcommand{\to}{\rightarrow}
\newcommand{\eqdef}{\mathrel{\stackrel{\rm def}=}}
\newcommand{\iid}{\mathrel{\stackrel{\rm i.\,i.\,d.}\sim}}
\newcommand{\const}{\mathrm{const}}
\usepackage[inline]{enumitem}

% вместо горизонтальной делаем косую черточку в нестрогих неравенствах
\renewcommand{\le}{\leqslant}
\renewcommand{\ge}{\geqslant}
\renewcommand{\leq}{\leqslant}
\renewcommand{\geq}{\geqslant}

\usepackage[backend=biber,bibencoding=utf8,sorting=nty,maxcitenames=4,style=numeric-verb]{biblatex}

\addbibresource{lit.bib}
\usepackage{csquotes}

\usepackage{multicol}
\usepackage{enumitem}

\renewcommand{\rmdefault}{cmss}
%\renewcommand{\ttdefault}{cmss}
\usepackage{sfmath}

\usepackage{pgfplotstable}
  
\begin{document}
	
	\maketitle
	
	\section*{Задание 1} 
	
	Пусть представлены следующие данные:
	\[
	Y = \begin{pmatrix}
	6 \\
	12 \\
	24 \\
	\end{pmatrix}, \text{ } X = \begin{pmatrix}
	1 & 6 \\
	1 & 3 \\
	1 & 6
	\end{pmatrix}
	\]
	\begin{enumerate}[label=\alph*)]
		\item Оценивается следующая модель:
		\[Y = X\beta + u,\]
		где $u$ – случайная ошибка.
		
		Найдите оценки коэффициентов модели. Найдите прогноз модели.
		\item Нанесите выборку на диаграмму рассеяния. Постройте оценённую линию регрессии. 
		\item Определите направление корреляции между $X$ и $Y$.
		\item Найдите выборочное среднее и выборочную дисперсию неконстантного признака.  
		\item Найдите коэффициент детерминации данной регрессии. 
	\end{enumerate}

	\section*{Задание 2}
	\begin{enumerate}[label=\alph*)]
		\item Запишите уравнение парной регрессии. Запишите уравнение оценённой линии в этой регрессии. 
		\item Запишите уравнение множественной регрессии в общем виде. Под каждым элементом (матрицей или вектором) подпишите его размер.
		\item Является ли линейной регрессией следующая модель:
		$Y = \beta_0 + \beta_1X_1X_2X_3 + u$?
		\item Является ли линейной регрессией следующая модель:
		$Y = \beta_0 + \beta_1^2X_1 + \beta_2^3X_2 + u$?
		\item Является ли линейной регрессией следующая модель:
		$Y = \beta_0 + \beta_1X_1^5 + u$?
	\end{enumerate}
	
	\section*{Задание 3}
	\begin{enumerate}[label=\alph*)]
		\item Что такое мультиколлинеарность признаков? К чему она может привести? Что можно сделать для устранения этой проблемы? 
		\item Опишите (по смыслу, без формул), как регуляризация помогает справляться с переобучением линейной регрессии.
	\end{enumerate}

	\section*{Задание 4}
	\begin{enumerate}[label=\alph*)]
		\item Опишите своими словами, что такое машинное обучение. В чём основная цель машинного обучения?
		\item Какие три типа задач машинного обучения в зависимости от наличия целевой переменной выделяют? 
		\item Приведите пример задачи многоклассовой классификации. 
		\item Как происходит обучение алгоритма (модели) машинного обучения? 
		\item Чем функционал качества отличается от метрики качества?
		\item Что такое недообучение? На одном графике схематически изобразите недообученную и корректно обученную модель. 
	\end{enumerate}
	
\end{document}