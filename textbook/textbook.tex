\documentclass[11pt, a4paper]{extarticle}

%% Язык
\usepackage{cmap} % Поиск в PDF
\usepackage{mathtext} % Кириллица в формулах
\usepackage[T2A]{fontenc} % Кодировка
\usepackage[utf8]{inputenc} % Кодировка
\usepackage[english,russian]{babel} % Локализация, переносы

%% Шрифты

% Serif
\usepackage{euscript} % Шрифт Евклид
\usepackage{mathrsfs} % Шрифт для математики

% Sans-serif
%\renewcommand{\rmdefault}{cmss}
%\renewcommand{\ttdefault}{cmss}
%\usepackage{sfmath}

% Настройки для xelatex
%\usepackage{polyglossia} % Для выбора языка в xelatex
%\setmainlanguage{russian}
%\setotherlanguages{english}
% Ligatures=TeX is on by default
% https://tex.stackexchange.com/questions/323542/
%\setmainfont[Ligatures=TeX]{Cantarell}
%\newfontfamily{\cyrillicfonttt}{Times New Roman}
%\newfontfamily\cyrillicfont{Cantarell}[Script=Cyrillic]
%\setsansfont[Ligatures=TeX]{Cantarell}
%\newfontfamily\cyrillicfontsf{Cantarell}[Script=Cyrillic]
%\setmonofont{Courier New}
%\newfontfamily\cyrillicfonttt{Courier New}[Script=Cyrillic]

%% Математика
\usepackage{amsmath, amsfonts, amssymb, amsthm, mathtools}
\usepackage{icomma}

% Операторы
\DeclareMathOperator*\plim{plim}
\DeclareMathOperator{\sgn}{sign}
\DeclareMathOperator{\sign}{sign}
\DeclareMathOperator*{\argmin}{arg\,min}
\DeclareMathOperator*{\argmax}{arg\,max}
\DeclareMathOperator*{\amn}{arg\,min}
\DeclareMathOperator*{\amx}{arg\,max}
\DeclareMathOperator{\cov}{Cov}
\DeclareMathOperator{\Var}{Var}
\DeclareMathOperator{\Cov}{Cov}
\DeclareMathOperator{\Corr}{Corr}
\DeclareMathOperator{\pCorr}{pCorr}
\DeclareMathOperator{\E}{\mathbb{E}}
\let\P\relax
\DeclareMathOperator{\P}{\mathbb{P}}
\renewcommand{\le}{\leqslant}
\renewcommand{\ge}{\geqslant}
\renewcommand{\leq}{\leqslant}
\renewcommand{\geq}{\geqslant}

% Распределения
\newcommand{\cN}{\mathcal{N}}
\newcommand{\cU}{\mathcal{U}}
\newcommand{\cBinom}{\mathcal{Binom}}
\newcommand{\cPois}{\mathcal{Pois}}
\newcommand{\cBeta}{\mathcal{Beta}}
\newcommand{\cGamma}{\mathcal{Gamma}}

% Множества
\def \R{\mathbb{R}}
\def \N{\mathbb{N}}
\def \Z{\mathbb{Z}}

% Другое
\newcommand{\dx}[1]{\,\mathrm{d}#1} % Для интеграла: маленький отступ и прямая d
\newcommand{\ind}[1]{\mathbbm{1}_{\{#1\}}} % Индикатор события
\newcommand{\iid}{\mathrel{\stackrel{\rm i.\,i.\,d.}\sim}}
\newcommand{\const}{\mathrm{const}}

%% Изображения
\usepackage{graphicx}
\usepackage{caption}
\usepackage{subcaption}
\usepackage{physics}
\usepackage{wrapfig} % Обтекание рисунков и таблиц текстом
\usepackage{tikz}

%% Таблицы
\usepackage{array, tabularx, tabulary, booktabs}
\usepackage{longtable}  % Длинные таблицы
\usepackage{multirow} % Слияние строк в таблице

%% Cписки
\usepackage{multicol}
\usepackage{enumitem}

%% Гиперссылки
\usepackage{xcolor}
\usepackage{hyperref}
\definecolor{linkcolor}{HTML}{8b00ff}
\hypersetup{colorlinks = true,
			linkcolor = linkcolor,
			urlcolor = linkcolor,
			citecolor = linkcolor}

%% Выравнивание
\setlength{\parskip}{0.5em} % Расстояние между абзацами
\usepackage{geometry} % Поля
\geometry{
	a4paper,
	left=20mm,
	top=20mm,
	right=20mm}
\setlength{\parindent}{0cm} % Отступ (красная строка)
\linespread{1.0} % Интерлиньяж

%% Оформление

% Красивый серый фон
\usepackage{framed} 
\definecolor{shadecolor}{gray}{0.9}

% Код
% https://www.overleaf.com/learn/latex/code_listing#Code_styles_and_colours
\newcommand{\code}[1]{{\tt #1}}
\usepackage{listings}

\usepackage{xcolor}

\definecolor{codegreen}{rgb}{0,0.6,0}
\definecolor{codegray}{rgb}{0.5,0.5,0.5}
\definecolor{codepurple}{rgb}{0.58,0,0.82}
\definecolor{backcolour}{rgb}{0.95,0.95,0.92}

\lstdefinestyle{mystyle}{
	backgroundcolor=\color{white},   
	commentstyle=\color{codegreen},
	keywordstyle=\color{magenta},
	numberstyle=\tiny\color{codegray},
	stringstyle=\color{codepurple},
	basicstyle=\ttfamily\footnotesize,
	breakatwhitespace=false,         
	breaklines=true,                 
	captionpos=b,                    
	keepspaces=true,                 
	numbers=left,                    
	numbersep=5pt,                  
	showspaces=false,                
	showstringspaces=false,
	showtabs=false,                  
	tabsize=2
}

\lstset{style=mystyle}

% Колонтитулы
\usepackage{fancyhdr}
\pagestyle{fancy}
\fancyhf{}
\fancyhead[L]{}
\fancyhead[R]{\thepage}

% Разделы и подразделы
\usepackage[sf, sl, outermarks]{titlesec}
\titleformat{\section}{\Large\bfseries\sffamily}{\thesection}{0.5em}{}
\titleformat{\subsection}{\large\sffamily}{\thesubsection}{0.5em}{}


% Содержание
\usepackage{tocloft}
\renewcommand{\cftsecfont}{\hspace{4.5em}\normalfont}
\renewcommand{\cftsubsecfont}{\hspace{5em}\normalfont}
\renewcommand{\cftsecpagefont}{\normalfont\hfill}
\renewcommand{\cfttoctitlefont}{\large\normalfont\hfill}
\renewcommand{\cftaftertoctitle}{\hfill}
\renewcommand{\cftsecleader}{\cftdotfill{\cftdotsep}}
\renewcommand{\cftsecafterpnum}{\hspace*{5.5em}\hfill}
\renewcommand{\cftsubsecafterpnum}{\hspace*{5.5em}\hfill}
\renewcommand{\cftsecaftersnum}{.}
\renewcommand{\cftsubsecaftersnum}{.}

%% Комментарии
\usepackage{comment}

%% To-do
\usepackage{todonotes}

%% Литература
\usepackage[backend = biber,
			bibencoding = utf8, 
			sorting = nty, 
			maxcitenames = 4,
			style = numeric-verb]{biblatex}
\addbibresource{lit.bib}
\usepackage{csquotes}

%% Заголовок
\title{Конспекты лекций для курса \\ <<Введение в машинное обучение и анализ данных на Python>>}
\author{Омелюсик Владимир}
\date{\today}

\begin{document}
	
	\maketitle
	\tableofcontents
	
	\section{Установка программного обеспечения}
	
	\subsection{Anaconda и Jupyter Notebook}
	%Установка программного обеспечения
	На нашем курсе мы будем использовать \lstinline|Python| -- современный и достаточно продвинутый язык программирования, который предоставляет много удобных инструментов для анализа данных и машинного обучения.
	Как и многие другие языки, \lstinline|Python| поставляется бесплатно, и установить его на свой компьютер может каждый желающий.
	Тем не менее, <<сырой>> \lstinline|Python| представляет из себя набор методов для работы с простыми объектами: числами, строками и так далее: мы можем написать программу в любом текстовом редакторе и попросить \lstinline|Python| выполнить её.
	Понятно, что для высокоуровневых задач, таких как машинное обучение и анализ данных, такой подход не очень удобен.
	
	Поэтому нам также потребуются дополнительные инструменты: во-первых, удобная \textit{среда}, в которой мы будем программировать. 
	Во-вторых, уже написанные методы для работы с анализом данных -- без них нам придётся программировать всё самостоятельно с нуля, а это займёт слишком много времени и усилий.
	Такие уже написанные кем-то методы обычно распространяются в виде \textit{библиотек}, которые можно установить, подключить и использовать.
	
	В этом курсе мы будем использовать \lstinline|Anaconda| -- потрясающий дистрибутив типа <<всё в одном>>.
	При установке \lstinline|Anaconda| на компьютер устанавливается \lstinline|Python|, среда программирования \lstinline|Spyder|, удобный инструмент для анализа данных \lstinline|Jupyter Notebook|, значительное число полезных библиотек и некоторые другие программы.
	
	\begin{shaded}
	\textbf{Инструкция по установке} \lstinline|Anaconda|.
	\begin{enumerate}
		\item Установщик \lstinline|Anaconda| можно загрузить с \href{https://www.anaconda.com/products/individual}{официального сайта}.
		Почти в самом низу страницы выбрите вашу операционную систему и скачайте Graphical Installer для последней доступной версии \lstinline|Python| (на момент написания это версия \textbf{3.8}).
		\item Запустите скачанный файл и установите \lstinline|Anaconda| в соответствии с инструкциями.
		\item После установки на Вашем компьютере появится приложение \lstinline|Anaconda Navigator|. Запустите его.
	\end{enumerate}
	\end{shaded}

	\lstinline|Anaconda Navigator| показывает входящие в \lstinline|Anaconda| программы в удобном виде.
	
	Непосредственно работать с анализом данных мы будем в программе \lstinline|Jupyter Notebook|.
	Эта программа проста в освоении и позволяет выполнять код кусочками, что оказывается очень удобным во многих задачах.
	Также она предоставляет простые средства для красивого оформления результатов исследований.
	Программа поставляется вместе с \lstinline|Anaconda|.
	
	\begin{shaded}
		\textbf{Инструкция по установке} \lstinline|Jupyter Notebook|.
		\begin{enumerate}
			\item В \lstinline|Anaconda Navigator| найдите \lstinline|Jupyter Notebook|.
			Если он не установлен, нажмите кнопку \lstinline|Install|. 
			\item Запустите \lstinline|Jupyter Notebook|.
		\end{enumerate}
	\end{shaded}

	После запуска \lstinline|Jupyter Notebook| вы, скорее всего, увидите длинный список папок.
	Это вид \textit{корневой папки} вашего компьютера.
	Её можно найти и обычным образом при использовании файловой системы.
	Например, на \lstinline|Windows| корневая папка находится по адресу \lstinline|C:\Users\imya_polzovatelya|.
	
	Рекомендуется создать в корневой папке подпапку и назвать её так, как вам нравится.
	Это будет наша рабочая папка, в которой мы будем хранить все материалы по ходу курса.
	Сделать это можно как в файловой системе (просто создать папку в корневой папке), так и в \lstinline|Jupyter Notebook| (кнопка \lstinline|New| $\to$ \lstinline|Folder|).
	
	Файлы, в которых мы будем работать, называются <<тетрадками>> (notebooks). 
	Чтобы создать новую тетрадку в той папке, которая отображается в интерфейсе \lstinline|Jupyter Notebook|, нажмите \lstinline|New| $\to$ \lstinline|Python 3|.
	Тетрадка автоматически откроется в новом окне.
	
	\subsection{Установка новых библиотек}
	Если вы обнаружили, что требуемая библиотека не установлена на вашем компьютере, то его можно установить следующим образом:
	\begin{shaded}
		\textbf{Инструкция по установке новых библиотек}.
		\begin{enumerate}
			\item Найдите на вашем компьютере программу \lstinline|Anaconda Prompt| (быстрее всего это можно сделать через поиск).
			Запустите её.
			Откроется интерфейс командной строки.
			\item Введите
			\begin{lstlisting}[language=bash]
				conda install package_name
			\end{lstlisting}
			где \lstinline|package_name| -- название пакета.
			Нажмите \lstinline|Enter|.
		\end{enumerate}
	\end{shaded}
	
	
	\section{Тема 1: Введение в анализ данных и статистику}
	%Введение в Python
	%Введение в анализ данных и статистику
	%Графический анализ данных и визуализация (pandas, matplotlib)
	
	\section{Тема 2: Элементы линейной алгебры}
	%Элементы линейной алгебры, numpy
	
	\section{Тема 3: Введение в машинное обучение}
	%Введение в машинное обучение
	
	\section{Тема 4: Линейные методы регрессии}
	%Теория по ЛР
	%Практика в sklearn
 	%Программирование собственной линейной модели
	
	\section{Тема 5: kNN}
	%Теория и практика kNN
	
	\section{Тема 6: Решающие деревья. Случайный лес}
	%Теория и практика
	
	\section{Тема 7: Нейронные сети}
	%Теория + программирование нейронной сети
	
	\section{Тема 8: Кластеризация}
	%k-Means, DBSCAN
	
\end{document}